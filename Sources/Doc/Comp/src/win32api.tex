\chapter{Using the Win32 API}
\label{win32api}

This chapter contains a short description of the Win32 API usage
in your Modula-2/Oberon-2 programs.

\section{Obtaining API documentation}

Your XDS package includes the Win32 API definition modules and the import
library, but no documentation on the API. You have to use
third-party books or the on-line documentation from the
Microsoft Win32 SDK, which is available on Microsoft
Developer Network CD-ROMs and along with some C compilers,
for instance, Microsoft Visual C++ or Watcom.

Unfortunately, we can not afford licensing the API
reference from Microsoft for adapting it to Modula-2
and distributing with XDS. We regret the inconvenience.

\section{Using the Win32 API in your programs}

The base Win32 API is collected in about 20 definition modules
residing in the \verb'DEF\WIN32' subdirectory of your \XDS{} installation.
There is also an utility which merges these modules, except the
module \verb'CommCtrl', into the single module \verb'Windows'
during installation. The installation program
also invokes the compiler to build the symbol files for all Win32 API modules.
To rebuild all these files anytime later,
change directory to \verb'xxx\DEF\WIN32', where \verb'xxx' is your
\XDS{} installation directory, and issue the following
commands:

\verb'    build\merge          ' to rebuild \verb'Windows.def' \\
\verb'    '{\tt \XC{}}\verb' =p =a windef.prj  ' to rebuild Win32 API symbol files

If you are just studying the Win32 API using SDK references or a
third party book, use the \verb'Windows' module.
Otherwise, you may wish to use separate modules.

\subsection{Windows.def: merged versus reexporting}

In versions prior to 2.31, you had to use either the merged
module \verb'Windows' {\it or} the separate modules, but
not both simultaneously, throughout all modules of your program,
to avoid compatibility problems. Starting from version 2.31, it is possible to
to build the \verb'Windows.def' file as a set of "reexport"
declarations, such as:

\verb'    TYPE ULONG = WinDef.ULONG;'\\
\verb'    CONST GetFileType = WinBase.GetFileType;'

This approach, proposed by Victor Sergeev, removes the restriction
on simultaneous usage of \verb'Windows.def' and other Win32 API definition
modules that resulted from compatibility problems.

To rebuild the \verb'Windows.def' file using this method,
issue the following command instead of \verb'build\merge':

\verb'    build\reexport'

{\bf Note:} As the new \verb'Windows.sym' file is about 10 percent
      larger, you may need to increase the \OERef{COMPILERHEAP}
      equation value in the compiler configuration file.

You may switch between the old "merge" and the new "reexport"
approaches to \verb'Windows.def' creation at any time, but note
that the \verb'CommCtrl' module imports from separate modules.

% !!! Exact list of modules with description.

\subsection{Reducing the size of Windows.def}

% !!! New in 2.40:

If are using the module \verb'Windows', but do not need some parts of the Win32 API
collected in it, you may build a reduced version by passing the name of the file
containing the list of modules you need to one of the utilitites \verb'merge' or
\verb'reexport'. One such list is in the \verb'DEF\WIN32\BUILD\MIN.LST' file that
does not include things like OLE and mulimedia. Its usage reduces the size of
\verb'Windows.def' file by about 30\% and considerably speeds up compilation of
modules importing the module 'Windows'.

\subsection{Building GUI applications}

If your program is a GUI application, toggle
the GUI option ON in the project file:

\verb'    +GUI'

\subsection{Declaring callbacks}

If you have to pass a procedure as a parameter to an API call (a window
procedure, for example), declare it as \verb'"StdCall"'
(see \ref{multilang:direct}):

\begin{verbatim}
    PROCEDURE ["StdCall"] MainDlgProc
    ( hwnd   : Windows.HWND
    ; msg    : Windows.UINT
    ; wparam : Windows.WPARAM
    ; lparam : Windows.LPARAM
    )        : Windows.BOOL;
\end{verbatim}

\ifcomment !!!
\subsection{Additional macros}

In order to reduce need for type casting, the following "macros" are
provided in the definition module \verb'OS2'
(they are implemented in the run-time library):

\begin{verbatim}
    MPFROMUSHORT    USHORT1FROMMP
    MPFROM2USHORT   USHORT2FROMMP
    MPFROMULONG     ULONGFROMMP

    MRFROMUSHORT    USHORT1FROMMR
    MRFROM2USHORT   USHORT2FROMMR
    MRFROMULONG     ULONGFROMMR
\end{verbatim}

Although these "macros" are declared in the module \verb'OS2', you are
requested to import the module \verb'OS2RTL' as well. More precisely,
the original IBM OS/2 Toolkit header files, which are used during C compilation
in case of XDS-C, do {\em not} contain these macros, so they are implemented in
the \verb'OS2RTL.h' header file supplied with XDS-C.
To have that header included into the generated C code, you need to
import the module \verb'OS2RTL'.
In XDS-x86, these macros are implemented as procedures which reside
in the XDS run-time library, so they will be linked in even if you do not
import \verb'OS2RTL', but importing it will ensure portability between
XDS-x86 and XDS-C.

\fi

\section{Using resources}

\subsection{Automatic resource binding}

In early \XDS{} versions, resources had to be bound to an executable
"manually", i.e. by invoking a resource compiler, each time after linking.
Starting from version 2.20, it is possible to specify any file in
the \See{{\tt!module} project file directive}{}{xc:project}
and iterate files by extensions in \See{makefile templates}
{}{xc:template}.

Thus, if your application uses resources, you may specify
binary resource files as modules in your project to have them
been automaticlly bound to the executable, provided that
the used \TRef{template}{}{xc:template} has adequate support:

\verb'    !module myapp.res'

\subsection{Header files}

\XDS{} package contains a resource compiler (XRC). Its current version
is compatible with traditional resource compilers found in C compiler
packages. Those compiles, however, have no idea about what
\mt{} is and work only with C header files. So does XRC.
Therefore, constants which identify resources have to be defined in
both C headers and \mt{} definition modules, which may
easily go out of sync.

To solve this problem, we suggest you to define resource identifiers
in header files and then use the H2D utility to convert these % ??? \ref{}
files to definition modules before compilation.

\section{Generic GUI application}

The \verb'SAMPLES\GENERIC' subdirectory of your XDS installation contains
a generic GUI application, which you can use as
a base for your own applications. The sample program uses the
standard menus and dialogs that most applications would use.
The source is designed to serve as a base for any GUI
application and was written in such a way that it can be
easily modified to handle any application specific commands.

