\chapter {Linking your program}
\label{link}

 Once you have the object modules for your program produced by the
 compiler, you have to use a linker to link them together with
 XDS runtime libraries and create an executable (EXE or DLL) file.
 You may use XDS Linker (XLINK), described in this chapter,
 or a \See{third-party linker}{}{link:third-party}
 which accepts input object files and libraries in OMF
 or COFF format.

 The compiler may invoke a linker for you seamlessly (see \ref{link:seamless}).

\section{Starting the linker}

 In order to link your program you have to invoke the linker specifying
 its input on the command line:

\verb'    xlink [options] [@responsefile] {filename}'

 Linker input consists of options and file name parameters.
 Command line options have to start with either \verb'"-"' or \verb'"/"' character.
 Option names are case insensitive; some of them allow shortening.
 File name parameters are used to specify object, library, and resource
 files to be linked.

 {\bf Example:}

 \verb'    xlink /sys=c /name=hello hello.obj d:\xds\lib\x86\libxds.lib'

\subsection{Using response files}

 Instead of specifying all linker input on the command line, you may put
 some options and file name parameters into a response file. You may combine
 response files with options and parameters on the command line.
 Options in a response file have to be separated by any whitespace characters
 (including line breaks).

 When you invoke the linker, use the following syntax:

\verb'    xlink {@responsefile}'

 where \verb'responsefile' is the name of the response file.

 The "\verb'@'" symbol indicates that the file is a response file. If the file is not
 in the working directory, specify the path for the file as well as the file
 name. If the linker cannot find a response file, it stops linking.

 You can begin using a response file at any point on the linker command
 line or in another response file.

 Options can appear anywhere in a response file. If an option is not
 valid, the linker generates an error message and stops linking.

 The semi-colon character (\verb'";"') at the beginning of a line forces
 the linker to treat it as a comment and ignore the whole line.

\subsubsection{Sample response file}

\begin{verbatim}
    ; XLINK response file
    /sys=VIO
    /stack=100000
    /name=runme.exe
    commands.obj
    echo.obj
    genecho.obj
    nodes.obj
    types.obj
    runme.obj
    d:\xds\lib\x86\xstart.lib
    d:\xds\lib\x86\libxds.lib
    d:\xds\lib\x86\os2face.lib
\end{verbatim}

\subsection{Seamless linking}
\label{link:seamless}

 The compiler, when invoked in the \See{PROJECT mode}{}{xc:modes:project},
 may automatically produce a response file and invoke the linker.
 This is the default behaviour after \XDS{} installation.

\havetowrite more in readme files. !!!

\section{Input and output}

\subsection{Specifying input files}

 To specify an input object, library, or resource file,
 pass its name to \verb'xlink'
 as a command line parameter or include its name into a response file.
 If a specified file name has no extension, .OBJ is assumed, so ensure
 that you specify library and resource files with extensions.

% !!! How the linker searches for files?

\subsection{Specifying output format}

 XLINK supports two output formats - Linear Executable Module Format (LX), used in OS/2,
 and Portable Executable Format (PE), used in Windows.
 Default is a format native to the host system. In order to explicitly specify the output
 format, use the option /IMAGEFORMAT=format, where format can be PE or LX.

 The linker may produce executable modules (EXE) and dynamic link libraries
 (DLL) in both LX and PE formats.
 An EXE file can be executed directly, for instance, by typing its
 name at the command prompt. In contrast, a DLL is executed when
 it is loaded by other processes, and cannot be invoked independently.

\subsection{Producing an executable module}
\label{link:inout:exetype:exe}

 The linker produces executable (EXE) files by default.

 To reduce the size of the EXE file and improve its performance, use the
 following options:

% !!! BASE should always be 10000 under OS/2

\begin{itemize}
\item /ALIGN=value to set the alignment factor in the output file. Set
       value to smaller factors to reduce the size of the executable, and to
       larger factors to reduce load time for the executable. By default, the
       alignment is set to 0x200.
\item /BASE=value to specify the load address for the executable.
       By default, the load address is set to 0x00010000 for LX format
       to 0x00400000 for PE format.
\item /SMART to detect unreachable code and omit it in the output file.
\item /FIXED to omit relocation information, which can result in a smaller
      executable.
\end{itemize}

  Use the /NAME option to specify an output file name. If you do not specify an
  extension, the linker will automatically append the extension \verb'.EXE' to
  the name you provided. If you omit the /NAME option,
  the linker will produce an .EXE file named after first .OBJ file specified.

  Use the /ENTRY option to explicitly set an entry point for the executable and
  the /NOENTRY option to set no entry point.

\subsection{Producing a dynamic link library}
\label{link:inout:exetype:dll}

 A dynamic link library (DLL) contains executable code for common
 functions, just as an ordinary library (.LIB) file does. However, when you
 link with a DLL (directly or using an \TRef{import library}{dll:using:loadtime}),
 the code in the DLL is not copied into the executable (EXE) file. Instead,
 only the import definitions for DLL functions are copied. At run time,
 the dynamic link library is loaded into memory, along with the EXE file.

 To produce a DLL as output, use the /DLL option. If you are using
 an export definition file (.EDF), specify its name in that same option
 as /DLL=edfname. For more information on using .EDF files, see section
 \ref{link:EDF}.

 To reduce the size of the DLL and improve its performance, use the
 following options:

\begin{itemize}
\item  /ALIGN=value to set the alignment factor in the output file. Set
       value to smaller factors to reduce the size of the executable, and to
       larger factors to reduce load time for the executable. By default, the
       alignment is set to 0x200.
\item  /BASE=value to specify the load address for the DLL.
       By default, the load address is set to 0x00010000 for LX format
       to 0x00400000 for PE format.
\item  /SMART to detect unreachable code and omit it in the output file.
\end{itemize}

  Use the /EXPORT command line option or an
  \See{export definition file}{}{link:EDF} to explicitly specify the
  functions and data you want to make available to other executables.

  Once you have built a DLL, you may produce an executable that
  links to that DLL. This can be done in two ways:
\begin{description}
\item[{\bf Using the DLL itself.}] \mbox{} \\
      If you want to produce a PE executable, you can specify a DLL
      name on the command line as a file name parameter,
      but make sure that the specified extension is .DLL.
% ??? What about LX?
\item[{\bf Using an import library.}] \mbox{} \\
      Use the \TRef{XLIB utility}{xlib} to create an import library
      for one or more DLLs. See \ref{xlib:modes:import} for
      more information
\end{description}

  If you do not specify an extension for the output file name in the /NAME
  option, the linker automatically adds the extension .DLL to the name you
  provide. If you omit the /NAME option and there is no
  specification in any object or library file, the linker generates a .DLL file
  named after the first object file specified.

  Use the /ENTRY option to explicitly set DLL initialization routine
  and the /NOENTRY option to avoid setting any initialization
  routine.

\subsection{Including debug information}
\label{link:inout:gendebug}

 Use the /DEBUG option to include symbolic debug information into the output file,
 to enable debugging at the source level, profiling, and      % !!! \ref
 comprehensive \See{call history dump}{}{rts:history}.
 The linker will embed symbolic data and line
 number information in the output file. The linker recognizes debug information
 in two formats - HLL4 and CodeView. Output format is always the one used in
 input object and library files. If there are input files containing
 debug information in different formats, a warning message is produced and debug
 information is not included into the output file.

 {\bf Note:} Linking with /DEBUG increases the size of the output file.

\subsection{Generating a map file}
\label{link:inout:genmap}

 Specify the /MAP option to generate a map file, which lists object modules in
 your output file and symbol information. If you do not specify a name for the
 map file, the map file takes the name of the output file with extension .MAP.

\subsection{Linker return codes}

 The linker has the following return codes:

\ifonline
\begin{tabular}{ll}
\else
\begin{tabular}{ll}
\fi
\bf  Code & \bf Meaning \\
\hline
  0       & Linking completed successfully. The linker detected no errors. \\
  255     & The linker detected errors during the linking process.
\end{tabular}

\section{Linking with library files}

\subsection{Linking with static libraries}

 The linker uses static library (.LIB) files to resolve external references from
 the code in the object (.OBJ) files. When the linker finds a function or data
 reference in an .OBJ file that requires a module from a library, the linker
 links the module from the library to the output file (for more details see
 the \LORef{SMART} option).

\subsection{Linking with dynamic link libraries}

\subsubsection{Using the DLL itself}

  If you want to produce a PE executable then you can specify a DLL
  name on the command line as a file name parameter, but make sure that
  specified extension is .DLL.

\subsubsection{Using an import library}

  An import library contains information about procedures and data
  exported by the DLL. Use the XLIB utility to create
  an import library, as described in \ref{xlib:modes:import}.

  You may then use the import library with the linker, when you generate
  executables that reference entities from the DLL. Enter the import library name
  as a file name parameter on the command line or in a response file.

\section{Setting linker options}
\label{link:options}

\subsection{Setting options on the command line}

 You can specify options anywhere on the command line. Separate options
 with a space or tab character.

 For example, to link an object file with the /MAP option, enter:

\verb'    XLINK /MAP myprog.obj'

\subsection{Setting numeric arguments}
\label{link:set_option:set_numeric}

 Some linker options take numeric arguments. The linker accepts numeric arguments
 in special syntax. You can specify numbers in any of the following forms:

\ifonline
\begin{tabular}{ll}
\else
\begin{tabular}{lp{9 cm}}
\fi
  Decimal         & Any number not prefixed with x or 0x is a decimal number. \\
                  & For example, 1234 is a decimal number. \\
  Hexadecimal     & Any number prefixed with x or 0x is a hexadecimal number. \\
                  & For example, 0x1234 is a hexadecimal number. \\
  K or M suffixes & Any number (either decimal or hexadecimal) is multiplied \\
                  & by 1024 if it is ended by suffix K and by 1048576 (1024 * 1024) \\
                  & if it is ended by suffix M. For example, 1K is 1024 and \\
                  & 1M is 1048576. \\
\end{tabular}

\subsection{Options reference}

\ifonline\else
\begin{tabular}{ll}
\bf Option & \bf Meaning \\
\hline
\LORef{ALIGN} & Alignment factor \\
\LORef{BASE} & Load address \\
\LORef{DEBUG} & Include debug info \\
\LORef{DLL} & Produce dynamic link library \\
\LORef{ENTRY} & Entry point \\
\LORef{NOENTRY} & No entry point \\
\LORef{EXPORT} & Define exported entities \\
\LORef{EXTRAALIGNMENT} & Align at 16-bytes boundary \\
\LORef{FIXED} & Omit relocation information \\
\LORef{HEAPCOMMIT} & Heap commit size \\
\LORef{HEAPSIZE} & Heap size \\
\LORef{IMAGEFORMAT} & Output file format \\
\LORef{MAP} & Produce map file \\
\LORef{NAME} & Output file name \\
\LORef{SMART} & Enable smart linking \\
\LORef{NOSMART} & Disable smart linking \\
\LORef{STACKCOMMIT} & Stack commit size \\
\LORef{STACKSIZE} & Stack size \\
\LORef{STUB} & Stub file \\
\LORef{SYS} & Application type \\
\LORef{WARN} & Warnings severity level
\end{tabular}
\fi

\ifonline\else
\begin{description}
\fi

\LinkOpt{ALIGN}{Alignment factor}
\begin{tabular}{ll}
\bf Syntax:  & \tt /ALIGN=factor \\
\bf Default: & \tt /ALIGN=512
\end{tabular}

 Use the /ALIGN option to set the alignment factor in the output EXE or DLL file.

 The alignment factor determines where pages in an executable (EXE or DLL) file
 start. From the beginning of the file, start of each page is aligned at
 a multiple (in bytes) of the alignment factor. The alignment factor must
 be a power of 2, from 1 to 4096.

 The default alignment is 512 bytes.

\LinkOpt{BASE}{Load address}
\begin{tabular}{ll}
\bf Syntax: & \tt /BASE=address \\
\bf Default:& {\tt /BASE=0x00010000} for LX format \\
            & {\tt /BASE=0x00400000} for PE format
\end{tabular}

 Use the /BASE option to specify the preferred load address for the module.

\LinkOpt{DEBUG}{Include debug info}
\begin{tabular}{ll}
\bf Syntax:  & \tt /D[EBUG] \\
\bf Default: & do not include debug info
\end{tabular}

 Use the /DEBUG option to include debug information in the output file,
 enabling it to be debugged at the source level.
 The linker will embed symbolic data
 and line number information in the output file.

 {\bf Note:} Linking with /DEBUG increases size of the output file.

\LinkOpt{DLL}{Produce dynamic link library}
\begin{tabular}{ll}
\bf Syntax:  & \tt /DLL[=edfname] \\
\bf Default: & build an EXE file
\end{tabular}

 Use the /DLL option to identify the output file as a dynamic link library (.DLL).
 To use export definition file (.EDF) specify its name after the equals sign:
 /DLL=edfname. For more information on producing DLLs, see \ref{link:inout:exetype:dll}.

 If you do not specify /DLL the linker will produce an EXE file.

\LinkOpt{ENTRY}{Entry point}
\begin{tabular}{ll}
\bf Syntax:  & \tt /ENTRY=identifier \\
\bf Default: & use the default entry point
\end{tabular}

 Use the /ENTRY option to explicitly specify an entry point for the EXE module or an
 initialization routine for the .DLL module. If there is an entry point in any object
 or library file being linked, it is ignored and a warning message is issued.

\LinkOpt{NOENTRY}{No entry point}
\begin{tabular}{ll}
\bf Syntax:  & \tt /NOENTRY \\
\bf Default: & use the default entry point
\end{tabular}

 Use the /NOENTRY option to avoid setting any entry point or initialization routine.

\LinkOpt{EXPORT}{Define exported entities}
\begin{tabular}{ll}
\bf Syntax:  & \tt /EXP[ORT]=external[.ordinal][=internal],... \\
\bf Default: & export all
\end{tabular}

 Use the /EXPORT option to specify exported data and functions on the command line.

\verb'    xlink . . . /EXPORT=Entry.1=Main_Entry,Main_BEGIN.2'

\LinkOpt{EXTRAALIGNMENT}{Align at 16-bytes boundary}
\begin{tabular}{ll}
\bf Syntax:  & \tt /EXTRA[ALIGNMENT] \\
\bf Default: & do not align at 16-bytes boundary
\end{tabular}

 Use the /EXTRAALIGNMENT option to align all at a 16-bytes boundary, if necessary.

\LinkOpt{FIXED}{Omit relocation information}
\begin{tabular}{ll}
\bf Syntax:  & \tt /FIXED \\
\bf Default: & include relocation information
\end{tabular}

 Use the /FIXED option to force the linker to omit relocation information
 in the output file.

\LinkOpt{HEAPCOMMIT}{Heap commit size}
\begin{tabular}{ll}
\bf Syntax:  & \tt /HEAPCOMMIT=number \\
\bf Default: & \tt /HEAPCOMMIT=0x00001000
\end{tabular}

 Use the /HEAPCOMMIT option to specify the heap commit size of your program.

\LinkOpt{HEAPSIZE}{Heap size}
\begin{tabular}{ll}
\bf Syntax:  & \tt /HEAP[SIZE]=number \\
\bf Default: & \tt /HEAPSIZE=0x00002000
\end{tabular}

 Use the /HEAPSIZE option to specify the maximum heap size of your program.

\LinkOpt{IMAGEFORMAT}{Output file format}
\begin{tabular}{ll}
\bf Syntax:  & \tt /IMAGE[FORMAT]=format \\
\bf Default: & {\tt /IMAGE[FORMAT]=LX} under OS/2 \\
             & {\tt /IMAGE[FORMAT]=PE} under Windows \\
\end{tabular}

 Use the /IMAGEFORMAT option to specify output executable file format.
 Format can be PE or LX. For more information, see \ref{link:inout:exetype:exe}.

\LinkOpt{MAP}{Produce map file}
\begin{tabular}{ll}
\bf Syntax:  & \tt /M[AP][=filename] \\
\bf Default: & do not generate map
\end{tabular}

  Use the /MAP option to generate map file. For more information see
  \ref{link:inout:genmap}.

\LinkOpt{NAME}{Output file name}
\begin{tabular}{ll}
\bf Syntax:  & \tt /NAME=filename \\
\bf Default: & use the first linked file name
\end{tabular}

  Use the /NAME option to explicitly specify the output file name. If
  \verb'filename' is not specified, the linker will name the
  output file after the first linked file.

\LinkOpt{SMART}{Enable smart linking}
\begin{tabular}{ll}
\bf Syntax:  & \tt /SMART \\
\bf Default: & \tt /SMART
\end{tabular}

 Use the /SMART option to enable smart linking for subsequent files.

\LinkOpt{NOSMART}{Disable smart linking}
\begin{tabular}{ll}
\bf Syntax:  & \tt /NOSMART \\
\bf Default: & \tt /SMART
\end{tabular}

 Use the /NOSMART option to disable smart linking for subsequent files.

\LinkOpt{STACKCOMMIT}{Stack commit size}
\begin{tabular}{ll}
\bf Syntax:  & \tt /STACKCOMMIT=number \\
\bf Default: & \tt /STACKCOMMIT=0x00002000
\end{tabular}

 Use the /STACKCOMMIT option to specify the stack commit size of your program.

\LinkOpt{STACKSIZE}{Stack size}
\begin{tabular}{ll}
\bf Syntax:  & \tt /STACK[SIZE]=number \\
\bf Default: & \tt /STACKSIZE=0x00010000
\end{tabular}

 Use the /STACKSIZE option to specify the maximum stack size of your program.

\LinkOpt{STUB}{Stub file}
\begin{tabular}{ll}
\bf Syntax:  & \tt /STUB=filename \\
\bf Default: & no stub file % ???
\end{tabular}

 Use the /STUB option to specify the stub file for your program.

\LinkOpt{SYS}{Application type}
\begin{tabular}{ll}
\bf Syntax:  & \tt /SYS=C | W [,number [.number]] \\
\bf Default: & \tt /SYS=C,4.0
\end{tabular}

 Use the /SYS option to force the linker to create console or windows application
 and specify major and minor OS version.

\LinkOpt{WARN}{Warnings severity level}
\begin{tabular}{ll}
\bf Syntax:  & \tt /WARN=(0|1|2|3) \\
\bf Default: & \tt /WARN=2
\end{tabular}

 Use the /WARN option to set the severity level of warnings the linker produces and
 that causes the warning count to increment. See \ref{xlink:errors}
 for the description of error and warning messages.

 You can set the severity levels from 0 up to 3.

\ifonline\else
\end{description}
\fi

\section{Using export definition files}
\label{link:EDF}
\index{export definition files}

{\em Export definition files (EDF)} are used to specify which
procedures and variables are to be visible outside a DLL.
EDF also allows arbitrary external names and ordinal numbers
to be assigned to those procedures and variables.

\subsection{Export definition file syntax}
\label{link:EDF:syntax}

 An .EDF file begins with the LIBRARY section:

\verb'     LIBRARY Identifier["." Identifier]'

 where \verb'Identifier' is defined as follows:

\verb!     identifier = (letter | "_") { letter | digit | "_" | "'")!

 LIBRARY section is optional and specified library name is unused.
 After it, the DESCRIPTION section goes:

\verb!     DESCRIPTION [ '"' Text '"' ]!

 DESCRIPTION section is also optional, but if it is present, the
 specified \verb'Text' will be written to the output file.

 After the DESCRIPTION section, the EXPORTS section goes:

\verb'    EXPORTS' \\
\verb'        { External [ "=" Internal ] [ "@" Ordinal ] }'

 The EXPORTS section is used to define the names of procedures and
 variables exported from the DLL.

\begin{description}
\item[{\tt External}] \mbox{} \\
      an external name of an exported object (procedure or variable), i.e.
      the name which has to be used in other executable modules
      to reference that object.
\item[{\tt Internal}] \mbox{} \\
      an internal name of that object, i.e. the name
      under which it appears in an object file. By default, the linker
      assumes that it is the same as \verb'External'.
\item[{\tt Ordinal}] \mbox{} \\
      a decimal number specifying ordinal position of the object in the
      module definition table. If it is not specified, the linker will
      assign an ordinal number automatically.
\end{description}

 The EXPORTS section is optional. % ???

\section{Linker error and warning messages}
\label{xlink:errors}

 0\\
 Unable to open file {\it file name}\\
 {\bf Explanation:} The linker can not open a file. It may be locked by
 another process or does not exist, path may be invalid or there is no
 enough room on used drive.\\
 {\bf Action:} Check possible reasons and relink.

 1\\
 Unable to write file {\it file name}\\
 {\bf Explanation:} The linker can not write a file. Possible reason is
 lack of enough room on used drive.\\
 {\bf Action:} Check it and relink.

 2\\
 Unable to read file {\it file name}\\
 {\bf Explanation:} \\
 {\bf Action:}

 3\\
 Insufficient memory\\
 {\bf Explanation:} There is no enough memory to link successfully.\\
 {\bf Action:} Close some applications and relink.

 5\\
 Invalid system {\it argument}\\
 {\bf Explanation:} Option /SYS is set to invalid value (different from 'W'
 or 'C').\\
 {\bf Action:} Check option setting validity and relink.

 6\\
 Invalid OS version {\it argument}\\
 {\bf Explanation:} OS version is set to invalid value (see option /SYS).\\
 {\bf Action:} Check option setting validity and relink.

 7\\
 Invalid number {\it argument}\\
 {\bf Explanation:} {\it Argument} is not valid number (see
 \ref{link:set_option:set_numeric}). \\
 {\bf Action:} Check argument and relink.

 8\\
 Invalid parameter - {\it argument}\\
 {\bf Explanation:} Specified parameter is not valid.\\
 {\bf Action:} Check parameter and relink.

 9\\
 Invalid /ENTRY option\\
 {\bf Explanation:} Entry point name is not specified in option /ENTRY setting.\\
 {\bf Action:} Check the specification and relink.

 10\\
 Invalid {\it option} value - {\it value}\\
 {\bf Explanation:} {\it Option} is set to invalid {\it value}.\\
 {\bf Action:} Check setting and relink.

 11\\
 Unrecognized option {\it argument} - option ignored\\
 {\bf Explanation:} Specified option is not recognized by the linker and
 ignored.\\
 {\bf Action:} Check option setting and relink.

 12\\
 No file(s) specified\\
 {\bf Explanation:} There are no files to link.\\
 {\bf Action:} Check and relink.

 13\\
 No DLL initializing routine\\
 {\bf Explanation:} DLL initializing routine is not specified
 or /NOENTRY option is set and there is no export from created library.\\
 {\bf Action:} Check the specification or setting and relink.

 14\\
 No program entry point\\
 {\bf Explanation:} Program entry point is not specified or /NOENTRY option
 is set.\\
 {\bf Action:} Check the specification or setting and relink.

 15\\
 Name {\it name} was declared in {\it module} and in {\it module}\\
 {\bf Explanation:} There are declarations of {\it name} in two various
 modules\\
 {\bf Action:} Investigate the reasons, correct and relink.

 16\\
 Name {\it name} was redeclared in {\it module}\\
 {\bf Explanation:} {\it Name} was redeclared in {\it module}.\\
 {\bf Action:} Investigate the reasons, correct and relink.

 17\\
 Illegal RES file {\it file name} - must be 32-bit Microsoft format\\
 {\bf Explanation:} Specified resource file consists of 32-bit Microsoft
 format header but its format is not 32-bit Microsoft format.\\
 {\bf Action:} Check resource file format and relink.

 18\\
 Illegal file format - {\it file name}\\
 {\bf Explanation:} Specified file format is not recognized by the liker.\\
 {\bf Action:} Check the specified file format and relink.

 19\\
 Empty file {\it file name}\\
 {\bf Explanation:} Specified file is empty.\\
 {\bf Action:} Check the specified file and relink.

 20\\
 File {\it file name} too long\\
 {\bf Explanation:} There is no enough memory to read specified file.\\
 {\bf Action:} Komu seichas legko?

 21\\
 Duplicate definition for export name {\it name}\\
 {\bf Explanation:} There is duplicate export definition for specified name.\\
 {\bf Action:} Check command line, .EDF and .OBJ file export specification
 and relink.

 22\\
 Entry point from {\it source} ignored, used from {\it source}\\
 {\bf Explanation:} There are two entry point specification, one of
 them is ignored and the other is used.\\
 {\bf Action:} To be sure, check entry point specifications and relink if
 it is needed.

 23\\
 File {\it file name} - illegal CPU type {\it type}\\
 {\bf Explanation:} Specified file has Common Object File Format and its
 CPU type is not i386 or UNKNOWN.\\
 {\bf Action:} Check the specified file and relink.

 24\\
 File {\it file name} - cannot initialize BSS segment {\it name}\\
 {\bf Explanation:} Specified file consists of uninitialized data segment
 with initialized data.\\
 {\bf Action:} Check the specified file and relink.

 25\\
 File {\it file name} - invalid weak external {\it name}\\
 {\bf Explanation:} Specified file consists of reference to invalid weak
 exrenal name.\\
 {\bf Action:} Check the specified file and relink.

 26\\
 File {\it file name} - illegal symbol index {\it index}\\
 {\bf Explanation:} Specified file cinsists of illegal symbol index.\\
 {\bf Action:} Check the specified file and relink.

 27\\
 File {\it file name} - unsupported fixup type {\it type}\\
 {\bf Explanation:} Specified file consists of fixup record of unsupported
 type.\\
 {\bf Action:} Check the specified file and relink.

 28\\
 File {\it file name} - invalid section {\it number}\\
 {\bf Explanation:} Specified file consists of section with invalid number.\\
 {\bf Action:} Check the specified file and relink.

 29\\
 File {\it file name} - bad storage class {\it number}\\
 {\bf Explanation:} Specified file consists of bad storage class object.\\
 {\bf Action:} Check the specified file and relink.

 30\\
 File {\it file name} - not expected end of file\\
 {\bf Explanation:} The end of specified file is not expected. Possible
 reason is the file is corrupt.\\
 {\bf Action:} Check the specified file and relink.

 31\\
 File {\it file name} ({\it raw} : {\it column}) - identifier expected\\
 {\bf Explanation:} The linker expects an identifier at specified location.\\
 {\bf Action:} Check the specified file and relink.

 32\\
 File {\it file name} ({\it raw} : {\it column}) - string isn't closed\\
 {\bf Explanation:} A string beginnig at specified location is not closed.\\
 {\bf Action:} Check the specified file and relink.

 33\\
 File {\it file name} ({\it raw} : {\it column}) - bad ordinal number\\
 {\bf Explanation:} Bad ordinal number is met at specified location.\\
 {\bf Action:} Check the specified file and relink.

 34\\
 File {\it file name} ({\it raw} : {\it column}) - ordinal number expected \\
 {\bf Explanation:} Ordinal number is expected at specified
 location.\\
 {\bf Action:} Check the specified file and relink.

 35\\
 File {\it file name} ({\it raw} : {\it column}) - end of file expected\\
 {\bf Explanation:} The end of file is expected at specified location.\\
 {\bf Action:} Check the specified file and relink.

 37\\
 File {\it file name} - unknown COMDAT length prefix {\it prefix}\\
 {\bf Explanation:} \\
 {\bf Action:}

 38\\
 File {\it file name} - invalid entry point\\
 {\bf Explanation:} There is invalid entry point specification.\\
 {\bf Action:} Check the specified file or entry point specification
 and relink.

 39\\
 File {\it file name} group {\it name} - local groups not supported\\
 {\bf Explanation:} The linker does not support local groups.\\
 {\bf Action:} Check the specified file and relink.

 40\\
 File {\it file name} - unsupported group type {\it type}\\
 {\bf Explanation:} There is unsupported group type in soecified file.\\
 {\bf Action:} Check the specified file and relink.

 41\\
 File {\it file name} - unknown common type {\it type}\\
 {\bf Explanation:} There is unknown common type in specified file.\\
 {\bf Action:} Check the specified file and relink.

 42\\
 File {\it file name} - too much data for segment {\it name}\\
 {\bf Explanation:} Specified segment lenght is not agree with lenght of
 specified segment data.\\
 {\bf Action:} Check the specified file and relink.

 43\\
 File {\it file name} - LINNUM segment not specified\\
 {\bf Explanation:} There is no segment for specified LINNUM record.\\
 {\bf Action:} Check the specified file and relink.

 44\\
 File {\it file name} - fixups in BSS segment {\it name}\\
 {\bf Explanation:} Fixup record for uninitialized data segment is illegal.\\
 {\bf Action:} Check the specified file and relink.

 45\\
 File {\it file name} - FIXUP without LEDATA\\
 {\bf Explanation:} There is no LEDATA for segment referenced by
 fixup record.\\
 {\bf Action:} Check the specified file and relink.

 46\\
 File {\it file name} - unsupported OMF record type\\
 {\bf Explanation:} There is unsupported OMF record type in specified file.\\
 {\bf Action:} Check the specified file and relink.

 47\\
 Illegal record length\\
 {\bf Explanation:} OMF record lenght has to be more then 2 bytes.\\
 {\bf Action:} Check the specified file and relink.

 48\\
 Record too long\\
 {\bf Explanation:} Specified OMF record lenght is too long.\\
 {\bf Action:} Check the specified file and relink.

 49\\
 File {\it file name} - unknown record type {\it type}\\
 {\bf Explanation:} There is unknown OMF record type in specified file.\\
 {\bf Action:} Check the specified file and relink.

 50\\
 File {\it file name} - internal name {\it name} not found\\
 {\bf Explanation:} Specified internal name is not found.\\
 {\bf Action:} Check the specified file and relink.

 51\\
 File {\it file name} - duplicate definition for export name {\it name}\\
 {\bf Explanation:} There is duplicate export definition for specified name.\\
 {\bf Action:} Check export definitions and relink.

 52\\
 File {\it file name} - unresolved segment {\it name}\\
 {\bf Explanation:} There is unresolved reference to specified segment.\\
 {\bf Action:} Check the specified file and relink.

 53\\
 Invalid fixup target\\
 {\bf Explanation:} Specified fixup target is invalid.\\
 {\bf Action:} ?

 54\\
 Invalid fixup for a flat memory model\\
 {\bf Explanation:} Specified fixup type is not valid for a flat memmory
 model.\\
 {\bf Action:} ?

 55\\
 Name {\it name} not found\\
 {\bf Explanation:} There is unresolved reference to specified name.\\
 {\bf Action:} ?

 56\\
 Name {\it name} not found (referenced in  {\it module}\\
 {\bf Explanation:} There is unresolved reference to specified name in
 specified file.\\
 {\bf Action:} Check the specified file and relink.

 57\\
 Invalid entry point target\\
 {\bf Explanation:} Entry point specification is invalid (has bad target).\\
 {\bf Action:} Check a file consisting of entry point specification and
 relink.

 58\\
 Illegal stub file {\it file name}\\
 {\bf Explanation:} Specified stub file is illegal.\\
 {\bf Action:} Check setting of /STUB option and specified stub file and
 relink.

 61\\
 File {\it file name} - unhandled EXE type or invalid EXE\\
 {\bf Explanation:} Specified DLL has unhandled or invalid format.\\
 {\bf Action:} Check the specified file and command line specifications
 and relink.

 62\\
 File {\it file name} - unable to find EXPORT section\\
 {\bf Explanation:} Specified dll has no export section.\\
 {\bf Action:} Check the specified file and command line specifications
 and relink.

 63\\
 Ivalid option {\it option} - option skipped\\
 {\bf Explanation:} set value of specified option is invalid and option setting
 is skipped.\\
 {\bf Action:} Check option setting and relink.

 64\\
 Ivalid image format {\it format}\\
 {\bf Explanation:} Specified image format is invalid.\\
 {\bf Action:} Check option /IMAGEFORMAT setting and relink.

 65\\
 Unable to mix debug information from modules - {\it module} (CV), {\it module} (HLL4)\\
 {\bf Explanation:} At least two modules with deffirent debug information
 formats are specified to link. The linker can not mix debug information.\\
 {\bf Action:} Check specified modules and relink.

 66\\
 Duplicate debug information for module {\it module}\\
 {\bf Explanation:} There is duplicate debug information for specified
 module.\\
 {\bf Action:} Check specified module and relink.

 67\\
 File {\it file name} - duplicate resource name\\
 {\bf Explanation:} There is duplicate resource name in specified resource
 file.\\
 {\bf Action:} Check the specified file and relink.

\section{Using third-party linkers}
\label{link:third-party}

You may use virtually any third-party linker accepting libraries and object files
in either OMF or COFF format (XLINK accepts both). Note, however, that the XDS
run-time library is in OMF format; COFF version is available on a special request.

The template file included into your XDS package contains support for
some widely used third-party linkers, such as Watcom's WLINK. Refer to the
Read Me First file from your on-line documentation.

\havetowrite more in readme files. !!!

