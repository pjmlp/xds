\chapter{\XDS{} Modula-2}\label{m2}\index{Modula-2}

This chapter covers details of the \XDS{} implementation of
the \mt{} language. In the standard
mode\footnote{When options \OERef{M2EXTENSIONS} and \OERef{M2ADDTYPES}
are OFF} \xds{} \mt{} complies with ISO 10514 (See
the statement of compliance and further details in \ref{m2:ISO}).
The compatibility rules are described in \ref{m2:ISO:comp}. The
differences between ISO \mt{} and the language described in
the 4th edition of Wirth's ``Programming in Modula-2'' 
\ifonline\else\cite{PIM}\fi % !!!
are listed in \ref{m2:ISO:features}. Language extensions are
described in \ref{m2:ext}.

\section{ISO Standard compliance}\label{m2:ISO}

\xds{} \mt{} partially complies with the requirements
of ISO 10514. The details of non-conformities are as follows:
\begin{itemize}
\item Not all libraries are available in the current release.
\item The current release may impose some restrictions
      on using new language features.
\end{itemize}
See Chapter \ref{limits} for further details.

\subsection{Ordering of declarations}\label{m2:dcl:order}

\xds{} \mt{} is a so-called `single-pass' implementation. It
means that all identifiers must be declared before use.
According to the International Standard this
{\em declare-before-use} approach is perfectly valid. The
alternative approach, ({\em declare-before-use-in-declarations}), can
be used in so-called `multi-pass' implementations.

A forward declaration must be used to allow forward references
to a procedure which actual declaration appears later in the
source text.

\Example
\begin{verbatim}
PROCEDURE a(x: INTEGER); FORWARD;
(* FORWARD declaration *)

PROCEDURE b(x: INTEGER);
BEGIN
  a(x-1);
END b;

PROCEDURE a(n: INTEGER);
(* proper procedure declaration *)
BEGIN
  b(n-1);
END a;
\end{verbatim}

To provide source compatibility between `single-pass' and
`multi-pass' implementations, the Standard requires that all
conforming `multi-pass' implementations accept and correctly
process the \verb'FORWARD' directive.

%\section{Implementation-defined features}
%
% The value of empty string is 0C.

\section{New language's features}\label{m2:ISO:features}

The  language  described  in the International Standard varies in
many details from the one described in Wirth's ``Programming
in Modula-2''\ifonline\else\cite{PIM}\fi. % !!!

The most important innovations are
\begin{itemize}
\item complex numbers
\item module finalization
\item exception handling
\item array and record constructors
\item four new system modules
\item standard library
\end{itemize}

{\bf Note:}  The system modules (except the module {\tt SYSTEM})
are not embedded in the compiler and are implemented as separate
modules.

\subsection{Lexis}

The ISO \mt{} has some new keywords
\ifonline\else (table \ref{table:m2:ISO:keywords},
page \pageref{table:m2:ISO:keywords})\fi{}
and pervasive identifiers
\ifonline\else (table \ref{table:m2:ISO:pervasive},
page \pageref{table:m2:ISO:pervasive})\fi,
and provides alternatives for some symbols
\ifonline\else(table \ref{table:m2:ISO:alt},
page \pageref{table:m2:ISO:alt})\fi.
It also introduces the syntax for source code directives (or pragmas):

\verb'    Pragma = "<*" pragma_body "*>"'

The Standard does not specify a syntax of \verb|pragma_body|.
In \XDS{}, source code directives are used for in-line option setting
and for conditional compilation.
See \ref{m2:pragmas:inline} for further details.

\begin{table}[bht]
\footnotesize
\begin{center}
\ifonline
\begin{tabular}{lll}
\else
\begin{tabular}{p{3.4cm}p{3.4cm}p{3.4cm}}
\fi
AND             & ARRAY                     &       BEGIN           \\
BY              & CASE                      &       CONST           \\
DEFINITION      & DIV                       &       DO              \\
ELSE            & ELSIF                     &       END             \\
EXIT            & \See{EXCEPT}{}{m2:ISO:exc} &       EXPORT          \\
\See{FINALLY}{}{m2:ISO:final} & FOR         &       \See{FORWARD}{}{m2:dcl:order}  \\
FROM            & IF                        &       IMPLEMENTATION  \\
IMPORT          & IN                        &       LOOP            \\
MOD             & MODULE                    &       NOT             \\
OF              & OR                        & \See{PACKEDSET}{}{m2:ISO:sets} \\
POINTER         & PROCEDURE                 & QUALIFIED             \\
RECORD          & \See{REM}{}{m2:ISO:rem}    & \See{RETRY}{}{m2:ISO:exc} \\
REPEAT          & RETURN                    & SET                   \\
THEN            & TO                        & TYPE                  \\
UNTIL           & VAR                       & WHILE                 \\
WITH            
\end{tabular}
\end{center}
\caption{Modula-2 keywords}\label{table:m2:ISO:keywords}
\end{table}

\begin{table}[hbtp] \tt
\footnotesize
\begin{center}
\begin{tabular}{p{5.0cm}p{5.0cm}}
ABS             & BITSET    \\
BOOLEAN         & CARDINAL  \\
CAP             & CHR             \\
CHAR            & COMPLEX (\ref{m2:ISO:complex})  \\
CMPLX (\ref{m2:ISO:complex})  &  DEC             \\
DISPOSE         & EXCL            \\
FALSE           & FLOAT     \\
HALT            & HIGH      \\
IM (\ref{m2:ISO:complex})       & INC             \\
INCL            & INT (\ref{m2:ISO:conv})      \\
INTERRUPTIBLE (\ref{m2:ISO:prot})  & INTEGER   \\
LENGTH (\ref{m2:ISO:strings})   & LFLOAT (\ref{m2:ISO:conv})         \\
LONGCOMPLEX (\ref{m2:ISO:complex})    & LONGREAL  \\
MAX             & MIN             \\
NEW             & NIL             \\
ODD             & ORD             \\
PROC            & PROTECTION (\ref{m2:ISO:prot})     \\
RE (\ref{m2:ISO:complex})       & REAL            \\
SIZE            & TRUE      \\
TRUNC           & UNINTERRUPTIBLE (\ref{m2:ISO:prot}) \\
VAL              \\
\end{tabular}
\end{center}
\caption{Modula-2 pervasive identifiers}\label{table:m2:ISO:pervasive}
\end{table}

\begin{table}[htbp]
\begin{center}
\begin{tabular}{clc}
\bf Symbol  & \bf Meaning    & \bf Alternative  \\
\hline
  \verb+[+  & left bracket   & \verb+(!+        \\
  \verb+]+  & right bracket  & \verb+!)+        \\
  \verb+{+  & left brace     & \verb+(:+        \\
  \verb+}+  & right brace    & \verb+:)+        \\
  \verb+|+  & case separator & \verb+!+         \\
  \verb+^+  & dereference    & \verb+@+         \\
\end{tabular}
\end{center}
\caption{Modula-2 alternative symbols}\label{table:m2:ISO:alt}
\end{table}

\subsection{Complex types}\label{m2:ISO:complex}
\index{Modula-2!complex types}
\index{COMPLEX}\index{complex numbers}

Types \verb'COMPLEX' and \verb'LONGCOMPLEX' can be used to represent complex
numbers.  These types differ in a the range and precision. The
\verb'COMPLEX' type is defined as a \verb'(REAL,REAL)' pair, while \verb'LONGCOMPLEX'
consists of a pair of \verb'LONGREAL' values.

There is no notation for a complex literal. A complex value can
be obtained by applying the standard function {\tt CMPLX} to two reals.
If both {\tt CMPLX} arguments are real constants the result is the
complex constant.

\verb'    CONST i = CMPLX(0.0,1.0);'

If both expressions are of the \verb'REAL' type, or if one is of the
\verb'REAL' type and the other is a real constant, the function returns
a \verb'COMPLEX' value. If both expressions are of the \verb'LONGREAL' type,
or if one is of the \verb'LONGREAL' type and the other is a real constant
the function returns a \verb'LONGCOMPLEX' value. The following table
summarizes the permitted types and the result type:
\begin{center}
\small \tt
\begin{tabular}{l|lll}
              & REAL     & LONGREAL    & real constant    \\
\hline
REAL          & REAL     & error       & COMPLEX          \\
LONGREAL      & error    & LONGCOMPLEX & LONGCOMPLEX      \\
real constant & COMPLEX  & LONGCOMPLEX & complex constant
\end{tabular}
\end{center}

Standard functions {\tt RE} and {\tt IM} can be used to obtain
a real or imaginary part of a value of a complex type. Both
functions have one formal parameter. If the actual parameter is of the \verb'COMPLEX'
type, both functions return a \verb'REAL' value; if the parameter is of
the \verb'LONGCOMPLEX' type, functions return a \verb'LONGREAL' value; otherwise the
parameter should be a complex constant and functions return a real
constant.

\verb'    CONST one = IM(CMPLX(0.0,1.0));'

There are four arithmetic binary operators for operands of a complex
type: addition (\verb|+|), subtraction (\verb|-|), multiplication
(\verb|*|), and division (\verb|/|). The following table indicates
the result of an operation for permitted combinations:

\begin{center}
\small \tt
\begin{tabular}{l|lll}
                 & COMPLEX  & LONGCOMPLEX & complex constant \\
\hline
COMPLEX          & COMPLEX  & error       & COMPLEX          \\
LONGCOMPLEX      & error    & LONGCOMPLEX & LONGCOMPLEX      \\
complex constant & COMPLEX  & LONGCOMPLEX & complex constant \\
\end{tabular}
\end{center}

There are two arithmetic unary operators that can be applied to
the values of a complex type: identity (\verb|+|) and negation
(\verb|-|). The result is of the operand's type.

Two complex comparison operators are provided for operands of
complex type: equality (\verb'=') and inequality (\verb'<>').

\Example
\begin{verbatim}
    PROCEDURE abs(z: COMPLEX): REAL;
    BEGIN
      RETURN RealMath.sqrt(RE(z)*RE(z)+IM(z)*IM(z))
    END abs;
\end{verbatim}

\subsection{Sets and packedsets}\label{m2:ISO:sets}
\index{Modula-2!PACKEDSET}

A set or packedset\footnote{Packedset types are innovated in
the Standard.} type defines a new elementary type whose set of values is
the power set of an associated ordinal type called the {\em base
type} of the set or packedset type.

\begin{verbatim}
    SetType       = SET OF Type;
    PackedsetType = PACKEDSET OF Type;
\end{verbatim}

The International Standard does not require a specific
representation for set types. Packedset types representation
has to be mapped to the individual bits of a
particular underlying architecture. The standard type \verb'BITSET' is a
predefined packedset type.

The current \xds{} implementation does not distinguish between
set and packedset types. A set of at least 256 elements can be
defined.

All set operators, namely union (\verb|+|), difference (\verb|-|),
intersection (\verb|*|), and symmetrical difference (\verb|/|), can
be applied to the values of both set and packedset types.

\begin{verbatim}
    TYPE
      CharSet = SET OF CHAR;
      ByteSet = PACKEDSET OF [-127..128];

    VAR
      letters, digits, alphanum: CharSet;
      neg, pos, zero : ByteSet;
     .  .  .
    letters  := CharSet{'a'..'z','A'..'Z'};
    digits   := CharSet{'0'..'9'};
    alphanum := letters + digits;

    neg := ByteSet{-127..-1}; pos := ByteSet{1..127};
    zero := ByteSet{-127..128}-neg-pos;
\end{verbatim}

\subsection{Strings}\label{m2:ISO:strings}
\index{Modula-2!string concatenation}

For operands of the string literal type, the string concatenation
operation is defined, denoted by the symbol \verb|"+"|. {\bf Note:}
a character number literal (e.g. \verb'15C') denotes a value of a literal
string type of length 1. The empty string is compatible with the
type \verb'CHAR' and has a value equal to the string terminator (\verb'0C').

\begin{verbatim}
    CONST
      CR = 15C;
      LF = 12C;
      LineEnd = CR + LF;
      Greeting = "hello " + "world" + LineEnd;
\end{verbatim}

The new standard function \verb'LENGTH' can be used to obtain the length
of a string value.

\verb'    PROCEDURE LENGTH(s: ARRAY OF CHAR): CARDINAL;'

\subsection{Value constructors}\label{m2:ISO:aggregates}
\index{Modula-2!value constructors}
\index{value constructors}

A value constructor is an expression denoting a value of an array
type, a record type, or a set type. In case of array
constructors and record constructors a list of values, known as
{\em structure components}, is specified to define the values of
components of an array value or the fields of a record value.
In case of a set constructor, a list of members is specified,
whose elements define the elements of the set value.

\begin{verbatim}
    ValueConstructor = ArrayValue
                     | RecordValue
                     | SetValue.
    ArrayValue  = TypeIdentifier "{"
                  ArrayComponent { "," ArrayComponent }
                  "}".
    ArrayComponent = Component [ BY RepeatCount ].
    Component   = Expression.
    RepeatCount = ConstExpression.
    RecordValue = TypeIdentifier "{"
                  Component { "," Component }
                  "}".
\end{verbatim}

Set constructors are described in PIM.

The total number of components of an array constructor must
be exactly the same as the number of array's elements (taking into
account repetition factors). Each component must be
assignment compatible with the array base type.

The number of components of a record constructor must be exactly
the same as the number of fields. Each component must be an assignment
compatible with the type of the field.

A special case is a record constructor for a record with variant
parts. If the $n$-th field is the tag field the $n$-th component
must be a constant expression. If there is no ELSE variant part
associated with the tag field, then the variant associated with the
value of expression should exist. If no variant is associated with
the value, then the fields of the ELSE variant part should be
included in the sequence of components.

The constructor's components may themselves contain lists of
elements, and such nested constructs need not specify a type
identifier. This relaxation is necessary for multi-dimensional
arrays, where the types of the inner components may be anonymous.

\Examples
\begin{verbatim}
    TYPE
      String = ARRAY [0..15] OF CHAR;
      Person = RECORD
        name: String;
        age : CARDINAL;
      END;
      Vector = ARRAY [0..2] OF INTEGER;
      Matrix = ARRAY [0..2] OF Vector;

    VAR
      string: String;
      person: Person;
      vector: Vector;
      matrix: Matrix;
       .  .  .
    BEGIN
       .  .  .
      string:=String{" " BY 16};
      person:=Person{"Alex",32};
      vector:=Vector{1,2,3};
      matrix:=Matrix{vector,{4,5,6},Vector{7,8,9}};
      matrix:=Matrix{vector BY 3};
\end{verbatim}

\subsection{Multi-dimensional open arrays}
\index{Modula-2!open arrays}

According to the International Standard, parameters of a
multi-dimensional open array type are allowed:

\begin{verbatim}
    PROCEDURE Foo(VAR matrix: ARRAY OF ARRAY OF REAL);
      VAR i,j: CARDINAL;
    BEGIN
      FOR i:=0 TO HIGH(matrix) DO
        FOR j:=0 TO HIGH(matrix[i]) DO
          ... matrix[i,j] ...
        END;
      END;
    END Foo;

    VAR a: ARRAY [0..2],[0..2] OF REAL;

    BEGIN
      Foo(a);
    END ...
\end{verbatim}

\subsection{Procedure type declarations}\label{m2:ISO:proc_dcl}

A procedure type identifier may be used in declaration of the type itself.
This feature is used in the Standard Library. See, for example,
modules {\tt ConvTypes} and {\tt WholeConv}.

\begin{verbatim}
    TYPE
      Scan = PROCEDURE (CHAR; VAR Scan);
      Func = PROCEDURE (INTEGER): Func;
\end{verbatim}

\subsection{Procedure constants}\label{m2:ISO:proc_const}

A constant expression may contain values of procedure types, or
structured values whose components are values of procedure types.
Procedure constants may be used as a mechanism for procedure
renaming. In a definition module it is possible to export a renamed
version of the imported procedure.

\Examples
\begin{verbatim}
    TYPE ProcTable = ARRAY [0..3] OF PROC;

    CONST
      WS = STextIO.WriteString;
      Table = ProcTable{Up,Down,Left,Right};
\end{verbatim}

\subsection{Whole number division}\label{m2:ISO:rem}

Along with {\tt DIV} and {\tt MOD} the International Standard
includes two additional operators for whole number division:
`\verb+/+' and {\tt REM}.

Operators {\tt DIV} and {\tt MOD} are defined for positive
divisors only, while `\verb+/+' and {\tt REM} can be used for
both negative and positive divisors.

The language exception {\tt wholeDivException} (See \ref{m2:ISO:exc})
is raised if:
\begin{itemize}
\item the second operand is zero (for all four operators)
\item the second operand of {\tt DIV} or {\tt MOD} is negative.
\end{itemize}

For the given {\tt lval} and {\tt rval}

\verb'    quotient  := lval / rval;'\\
\verb'    remainder := lval REM rval;'

the following is true (for all non-zero values of {\tt rval}):
\begin{itemize}
\item \verb'lval = rval * quotient + remainder'

\item the value of {\tt remainder} is either zero, or an integer of
the same sign as {\tt lval} and of a smaller absolute value than {\tt rval}.
\end{itemize}

For the given {\tt lval} and {\tt rval}

\verb'    quotient := lval DIV rval;'\\
\verb'    modulus  := lval MOD rval;'

the following is true (for all positive values of {\tt rval}):
\begin{itemize}
\item \verb'lval = rval * quotient + modules'

\item the value of {\tt modulus} is a non-negative integer
less than {\tt rval}.
\end{itemize}

Operations are exemplified in the following table:
\begin{center}
\begin{tabular}{c|c|c|c|c}
{\em op} & 31 {\em op} 10 & 31 {\em op} (-10) & (-31) {\em op} 10 & (-31) {\em op} (-10) \\
\hline
\verb+/+ &  3  & -3         &  -3  & 3          \\
\tt REM  &  1  & 1          &  -1  & -1         \\
\tt DIV  &  3  & exception  &  -4  & exception  \\
\tt MOD  &  1  & exception  &   9  & exception  \\
\end{tabular}
\end{center}

\subsection{Type conversions}\label{m2:ISO:conv}

The language includes the following type conversion
functions: {\tt CHR}, {\tt FLOAT}, {\tt INT}, {\tt LFLOAT}, {\tt
ORD}, {\tt TRUNC} and {\tt VAL}. The functions {\tt INT} and {\tt
LFLOAT} are not described in PIM.

All the type conversion functions (except {\tt VAL}) have a single
parameter and can be expressed in terms of the {\tt VAL}
function.
\begin{center}
\begin{tabular}{lll}
\bf Function    & \bf Parameter   & \bf Equals to       \\
\hline
\tt CHR(x)      & whole           & \tt VAL(CHAR,x)     \\
\tt FLOAT(x)    & real or whole   & \tt VAL(REAL,x)     \\
\tt INT(x)      & real or ordinal & \tt VAL(INTEGER,x)  \\
\tt LFLOAT(x)   & real or whole   & \tt VAL(LONGREAL,x) \\
\tt ORD(x)      & ordinal         & \tt VAL(CARDINAL,x) \\
\tt TRUNC(x)    & real            & \tt VAL(CARDINAL,x) \\
\end{tabular}
\end{center}

The function \verb'VAL' can be used to obtain a value of the specified
scalar type from an expression of a scalar type. The function has
two parameters. The first parameter should be a type parameter
that denotes a scalar type. If the type is a subrange type, the
result of \verb'VAL' has the host type of the subrange type, otherwise
it has the type denoted by the type parameter.

The second parameter should be an expression of a scalar type
and at least one of the restriction shall hold:
\begin{itemize}
\item the result type and the type of the expression are identical
\item both the result type and the type of the expression are whole or
      real
\item the result type or the type of the expression is a whole type
\end{itemize}

In the following table, $\surd$ denotes a valid combination of types
and $\times$ denotes an invalid combination:
\begin{center}
\begin{tabular}{l|ccccc} \hline
the type of & \multicolumn{5}{c}{the type denoted by the type parameter} \\
expression  & whole   & real     & \tt CHAR &\tt BOOLEAN  & enumeration \\
\hline
whole type  & $\surd$ & $\surd$  & $\surd$  & $\surd$  & $\surd$  \\
real type   & $\surd$ & $\surd$  & $\times$ & $\times$ & $\times$ \\
\tt CHAR    & $\surd$ & $\times$ & $\surd$  & $\times$ & $\times$ \\
\tt BOOLEAN & $\surd$ & $\times$ & $\times$ & $\surd$  & $\times$ \\
enumeration & $\surd$ & $\times$ & $\times$ & $\times$ & $\surd$  \\
\end{tabular}
\end{center}

An exception is raised if the value of \verb'x' is outside the range of
the type \verb'T' in the call {\tt VAL(T,x)}. If {\tt x} is of a real type,
the calls {\tt VAL(INTEGER,x)} and {\tt VAL(CARDINAL,x)} both
truncate the value of {\tt x}.

\subsection{NEW and DISPOSE}\label{m2:ISO:NEW}
\index{Modula-2!NEW and DISPOSE}
\index{NEW (M2)}\index{DISPOSE}
\index{memory management}

The standard procedures \verb'NEW' and \verb'DISPOSE' are back in the language.
Calls of \verb'NEW' and \verb'DISPOSE' are substituted by calls of {\tt
ALLOCATE} and {\tt DEALLOCATE} which should be visible at the
current scope. The compiler checks compatibility of these
substitution procedures with the expected formal type:

\verb'    PROCEDURE ALLOCATE(VAR a: ADDRESS; size: CARDINAL);'\\
\verb'    PROCEDURE DEALLOCATE(VAR a: ADDRESS; size: CARDINAL);'

As a rule, the procedures {\tt ALLOCATE} and {\tt DEALLOCATE} declared
in the module {\tt Storage} are used. These procedures are made
visible by including the import list:

\verb'    FROM Storage IMPORT ALLOCATE, DEALLOCATE;'

When language extensions are enabled, the procedures {\tt NEW} and
{\tt DISPOSE} can be applied to dynamic arrays. See \ref{m2:ext:NEW}
for further details.

See also the \OERef{STORAGE} option.

\subsection{Finalization}\label{m2:ISO:final}
\index{finalization}

A special mechanism called {\em finalization} is provided to
perform certain operations during program termination.

A module declaration contains an optional finalization body, which
is executed during program termination for static modules (See
\ref{m2:ISO:termination}) or dynamic module finalization.

\begin{verbatim}
    ModuleBody = [ BEGIN BlockBody
                 [ FINALLY BlockBody ] ] END
    BlockBody  = NormalPart
                 [ EXCEPT ExceptionalPart ].
    NormalPart = StatementSequence.
    ExceptionalPart = StatementSequence.
\end{verbatim}
{\bf Note:} the \verb'RETURN' statement can be used in a {\tt BlockBody}.

Consider the following example:

\begin{verbatim}
    MODULE Test;

     .  .  .

    VAR cid: StreamFile.ChanId;

    BEGIN
      StreamFile.Open(cid,"tmp",flags,res);
      Process(cid);
    FINALLY
      StreamFile.Close(cid);
    END Test
\end{verbatim}

If the {\tt Test} module is declared in a procedure block, then
the initialization body will be executed on a call of the
procedure, while the finalization body is executed upon return
from the procedure.

If the {\tt Test} module is a static module, its finalization
will be executed during program termination.

In any case, finalization bodies are executed in reverse
order with respect to their initializations.

In the following example, finalization of a local module
is used to restore context:

\begin{verbatim}
    VAR state: State;

    PROCEDURE Foo;

      MODULE AutoSave;
        IMPORT state, State;
        VAR save: State;
      BEGIN
        save:=state; (* save state *)
        state:=fooState;
      FINALLY
        state:=save; (* restore state *)
      END AutoSave;

    BEGIN
      ... process ...
    END Foo;
\end{verbatim}

The initialization part of the {\tt AutoSave} module will be
executed before any statement in the {\tt Foo} body and finalization
part will be executed directly before returning from a call of {\tt
Foo}.

\subsection{Exceptions}\label{m2:ISO:exc}
\index{exceptions}\index{Modula-2!exceptions}

An exception handling mechanism is now included in the language.
Both user-defined exceptions and language exceptions can be
handled. There is no special exception type; an exception is
identified by a pair: exception source value and cardinal
value. Two keywords (\verb'EXCEPT' and \verb'RETRY') are added to the
language. The essential part of exception handling is provided in
two system modules: {\tt EXCEPTIONS} and {\tt M2EXCEPTION}.

The {\tt EXCEPTIONS} module provides facilities for raising and
identifying the user-defined exceptions, for reporting their
occurrence, and for making enquiries concerning the execution
state of the current coroutine.

The {\tt M2EXCEPTION} module provides facilities for identifying
language exceptions that have been raised.

A procedure body, an initialization or finalization part of a module
body may contain an exceptional part.
\begin{verbatim}
    BlockBody  = NormalPart [ EXCEPT ExceptionalPart ].
    NormalPart = StatementSequence.
    ExceptionalPart = StatementSequence.
\end{verbatim}

Example:

\begin{verbatim}
    PROCEDURE Div(a,b: INTEGER): INTEGER;
    BEGIN
      RETURN a DIV b      (* try to divide *)
    EXCEPT
      RETURN MAX(INTEGER) (* if exception *)
    END Fly;
\end{verbatim}

When an exception is raised (explicitly or implicitly) the
`nearest' (in terms of procedure calls) exceptional part in the
current coroutine receives control. Each coroutine is
executed initially in the normal state. If an exception is raised, the
coroutine state switches to the exceptional state. If there is no
exceptional part, raising of an exception is a termination event
(See \ref{m2:ISO:termination}).

A procedure with an exceptional part is executed in the normal
state. The state is restored after block execution. A
procedure without an exceptional part is executed in the state of
the caller.

If an exception is raised in the state of exceptional execution
it is re-raised in the calling context. In this case finalization
of local modules and restoring protection (See \ref{m2:ISO:prot})
will not take place.

An additional statement (\verb'RETRY') can be used in the exceptional
part. Execution of the \verb'RETRY' statement causes the normal part to
be re-executed in the normal state.

Execution of the \verb'RETURN' statement in the exceptional part
causes switch to the normal state.

If neither \verb'RETURN' nor \verb'RETRY' was executed in the exceptional part,
the exceptional completion will occur. In this case after
finalization of local modules (if any) and restoring protection
state (if necessary), the exception will be re-raised.

\Example
\begin{verbatim}
    PROCEDURE Foo;
    BEGIN
      TryFoo(...);
    EXCEPT
      IF CanBeRepaired() THEN
        Repair;
        RETRY;  (* re-execute the normal part *)
      ELSIF CanBeProcessed() THEN
        Process;
        RETURN; (* exception is handled *)
      ELSE
        (* exception will be automatically re-raised *)
      END;
    END Foo;
\end{verbatim}

\subsection{The system module EXCEPTIONS}
\label{m2:ISO:exceptions}
\index{EXCEPTIONS}
\index{system modules!EXCEPTIONS}
\index{Modula-2!EXCEPTIONS}

The module {\tt EXCEPTIONS} provides facilities for raising user's
exceptions and for making enquiries concerning the current
execution state.

User-defined exceptions are identified uniquely by a pair
(exception source, number). When the source of a used-defined
exception is a separate module, it prevents the defined
exceptions of the module from being raised directly by other
sources. See e.g. the module {\tt Storage}.

\verb'    TYPE ExceptionSource;'

\index{ExceptionSource}
\index{EXCEPTIONS!ExceptionSource}
Values of the opaque type {\tt ExceptionSource} are used to
identify the source of exceptions raised;
they should be allocated before usage.

\verb'    TYPE ExceptionNumber = CARDINAL;'

\index{ExceptionNumber}
\index{EXCEPTIONS!ExceptionNumber}
Values of the type {\tt ExceptionNumber} are used to
distinguish between different exceptions of one source.

\verb'    PROCEDURE AllocateSource(VAR newSource: ExceptionSource);'

\index{AllocateSource}
\index{EXCEPTIONS!AllocateSource}
The procedure allocates an unique value of the type {\tt
ExceptionSource}. The procedure is normally called during
initialization of a module, and the resulting value is then
used in all calls of {\tt RAISE}. If an unique value cannot be
allocated the language exception {\tt exException} is raised
(See \ref{m2:m2exception}).

\begin{verbatim}
   PROCEDURE RAISE(source: ExceptionSource;
                   number: ExceptionNumber;
                  message: ARRAY OF CHAR);
\end{verbatim}

\index{RAISE}
\index{EXCEPTIONS!RAISE}
A call to \verb'RAISE' associates the given values of exception
\verb'source', \verb'number', and \verb'message'
with the current context and raises an exception.

The function {\tt CurrentNumber} can be used to obtain the
exception number for the current exception.

\begin{verbatim}
    PROCEDURE CurrentNumber
       (source: ExceptionSource): ExceptionNumber;
\end{verbatim}

\index{CurrentNumber}
\index{EXCEPTIONS!CurrentNumber}
If the calling coroutine is in the exceptional execution state
because of raising an exception from {\tt source}, the procedure
returns the corresponding number, and otherwise raises an
exception.

The procedure {\tt GetMessage} can be used to obtain the message
passed when an exception was raised. This may give further
information about the nature of the exception.

\verb'    PROCEDURE GetMessage(VAR text: ARRAY OF CHAR);'

\index{GetMessage}
\index{EXCEPTIONS!GetMessage}
If the calling coroutine is in the exceptional execution state, the
procedure returns the (possibly truncated) string associated with the
current context. Otherwise, in the normal execution state, it returns the
empty string.

\begin{verbatim}
    PROCEDURE IsCurrentSource
            (source: ExceptionSource): BOOLEAN;
\end{verbatim}

\index{IsCurrentSource}
\index{EXCEPTIONS!IsCurrentSource}
If the current coroutine is in the exceptional execution state
because of raising an exception from \verb'source',
the procedure returns \verb'TRUE', and \verb'FALSE' otherwise.

\verb'    PROCEDURE IsExceptionalExecution (): BOOLEAN;'

\index{IsExceptionalExecution}
\index{EXCEPTIONS!IsExceptionalExecution}
If the current coroutine is in the exceptional execution state
because of raising an exception, the procedure returns \verb'TRUE', and
\verb'FALSE' otherwise.

The following example illustrates the recommended form of a library
module and usage of procedures from {\tt EXCEPTIONS}.
\begin{verbatim}
    DEFINITION MODULE FooLib;

    PROCEDURE Foo;
    (* Raises Foo exception if necessary *)

    PROCEDURE IsFooException(): BOOLEAN;
    (* Returns TRUE, if the calling coroutine is in
       exceptional state because of the raising of
       an exception from Foo, and otherwise returns FALSE.
    *)

    END FooLib.

    IMPLEMENTATION MODULE FooLib;

    IMPORT EXCEPTIONS;

    VAR source: EXCEPTIONS.ExceptionSource;

    PROCEDURE Foo;
    BEGIN
      TryFoo(...);
      IF NOT done THEN
        EXCEPTIONS.RAISE(source,0,"Foo exception");
      END;
    END Foo;

    PROCEDURE IsFooException(): BOOLEAN;
    BEGIN
      RETURN EXCEPTIONS.IsCurrentSource(source)
    END IsLibException;

    BEGIN
      EXCEPTIONS.AllocateSource(source)
    END FooLib.
\end{verbatim}

If we want to distinguish the exceptions raised in the {\tt FooLib}
we will append an enumeration type and an additional
enquiry procedure in the {\tt FooLib} definition:
\begin{verbatim}
    TYPE FooExceptions = (fault, problem);

    PROCEDURE FooException(): FooExceptions;
\end{verbatim}
The {\tt FooException} procedure can be implemented as follows:
\begin{verbatim}
    PROCEDURE FooException(): FooExceptions;
    BEGIN
      RETURN VAL(FooExceptions,
                 EXCEPTIONS.CurrentNumber(source))
    END FooException;
\end{verbatim}

The {\tt Client} module illustrates the usage of the library module
\verb'FooLib':

\begin{verbatim}
    MODULE Client;

    IMPORT  FooLib, EXCEPTIONS, STextIO;

    PROCEDURE ReportException;
      VAR s: ARRAY [0..63] OF CHAR;
    BEGIN
      EXCEPTIONS.GetMessage(s);
      STextIO.WriteString(s);
      STextIO.WriteLn;
    END ReportException;

    PROCEDURE TryFoo;
    BEGIN
      FooLib.Foo;
    EXCEPT
      IF FooLib.IsFooException() THEN
        ReportException;
        RETURN; (* exception is handled *)
      ELSE
        (* Exception will be re-raised *)
      END
    END TryFoo;

    END Client.
\end{verbatim}

\subsection{The system module M2EXCEPTION}\label{m2:m2exception}
\index{M2EXCEPTION@{\bf M2EXCEPTION}}
\index{system modules!M2EXCEPTION}
\index{Modula-2!M2EXCEPTION}

The system module {\tt M2EXCEPTION} provides language exceptions
identification facilities. The language (which includes the
system modules) is regarded as one source of exceptions.

The module exports the enumeration type {\tt M2Exceptions},
used to distinguish language exceptions,
and two enquiry functions.

\begin{verbatim}
    TYPE
      M2Exceptions =
        (indexException,      rangeException,
         caseSelectException, invalidLocation,
         functionException,   wholeValueException,
         wholeDivException,   realValueException,
         realDivException,    complexValueException,
         complexDivException, protException,
         sysException,        coException,
         exException
        );
\end{verbatim}

\verb'    PROCEDURE IsM2Exception(): BOOLEAN;'

\index{IsM2Exception}
\index{M2EXCEPTION!IsM2Exception}
If the current coroutine is in the exceptional execution state
because of the raising of a language exception, the procedure returns
\verb'TRUE', and \verb'FALSE' otherwise.

\verb'    PROCEDURE M2Exception(): M2Exceptions;'

\index{M2Exception}
\index{M2EXCEPTION!M2Exception}
If the current coroutine is in the exceptional execution state
because of the raising of a language exception, the procedure returns the
corresponding enumeration value, and otherwise raises an
exception.

The following description lists all language exceptions
(in alphabetical order) along with the circumstances under which
they are detected. {\bf Note:} Compiler options
can be used to control detection of some exceptions
(See Chapter \ref{options}).
Detection of some exceptions is not required by the Standard, however
such exceptions can be detected on some platforms
(See Chapter \ref{limits}).
\begin{description}
\item[caseSelectException] \mbox{}
\index{caseSelectException}
\index{M2EXCEPTION!caseSelectException}

         Case selector is out of range and the \verb'ELSE' clause does not exist.

\item[coException] \mbox{}
\index{M2EXCEPTION!coException}

        The system module \See{{\bf COROUTINES}}{}{m2:ISO:coroutines}
        exceptions:
        \begin{itemize}
        \item RETURN from a coroutine other than the main coroutine
        \item size of the supplied workspace is smaller than the
              minimum required (See description of the procedure
              \TRef{{\tt NEWCOROUTINE}}{m2:ISO:coroutines})
        \item the caller is not attached to the source of
              interrupts (See description of the procedure
              \TRef{{\tt HANDLER}}{m2:ISO:coroutines})
        \item coroutine workspace overflow
        \end{itemize}

\item[complexDivException] \mbox{}
\index{M2EXCEPTION!complexDivException}

        Divide by zero in a complex number expression.

\item[complexValueException] \mbox{}
\index{M2EXCEPTION!complexValueException}

        Overflow in evaluation of a complex number expression.

\item[exException] \mbox{}
\index{M2EXCEPTION!exException}

        A system module \TRef{{\bf EXCEPTIONS}}{m2:ISO:exceptions}
        or {\bf M2EXCEPTION} exception:
        \begin{itemize}
        \item exception identity is enquired in the normal execution state
              (See {\tt CurrentNumber})
        \item exception identity enquiry to a wrong source
              (See {\tt CurrentNumber})
        \item no further exception source values can be allocated
              (See {\tt AllocateSource})
        \end{itemize}

\item[functionException] \mbox{}
\index{M2EXCEPTION!functionException}

        No \verb'RETURN' statement before the end of a function.

\item[indexException] \mbox{}
\index{M2EXCEPTION!indexException}

        Array index out of range.
        See options \OERef{CHECKINDEX} and \OERef{CHECKDINDEX}.

\item[invalidLocation] \mbox{}
\index{M2EXCEPTION!invalidLocation}

        Attempt to dereference \verb'NIL' or an uninitialized pointer.
        See the option \OERef{CHECKNIL}.

\item[protException] \mbox{}
\index{M2EXCEPTION!protException}

        The given protection is less restrictive than the current protection.

\item[rangeException] \mbox{}
\index{M2EXCEPTION!rangeException}

        Range exception (See the \OERef{CHECKRANGE} option):
        \begin{itemize}
        \item assignment value is out of range of the target's type
        \item structure component value is out of range
        \item expression cannot be converted to the new type
        \item value to be included/excluded is not of the
              base type of the set (See also the \OERef{CHECKSET} option)
        \item return value is out of range
        \item set value is out of range
              (See also the \OERef{CHECKSET} option)
        \item tag value is out of range (in a variant record).
        \end{itemize}

\item[realDivException] \mbox{}
\index{M2EXCEPTION!realDivException}

        Divide by zero in a real number expression.

\item[realValueException] \mbox{}
\index{M2EXCEPTION!realValueException}

        Overflow in evaluation of a real number expression.

\item[sysException] \mbox{}
\index{M2EXCEPTION!sysException}

        The system module {\bf SYSTEM} exceptions.
        {\bf Note:} All these exceptions are non-mandatory.
        \begin{itemize}
        \item invalid use of {\tt ADDADR}, {\tt SUBADR} or {\tt DIFADR}
        \item the result of {\tt MAKEADR} is out of the address range
        \item alignment problem with {\tt CAST}
        \item the result of {\tt CAST} is not a valid representation
              for the target type
        \end{itemize}

\item[wholeDivException] \mbox{}
\index{M2EXCEPTION!wholeDivException}

        Whole division exception:
        \begin{itemize}
        \item divided by zero in evaluation of a whole number
              expression
        \item the second operand of \verb'DIV' or \verb'MOD' is negative
              (See the \OERef{CHECKDIV} option)
        \end{itemize}

\item[wholeValueException] \mbox{}
\index{M2EXCEPTION!wholeValueException}

        Overflow in evaluation of a whole number expression.
\end{description}

\paragraph{An example of language exception handling}
\begin{verbatim}
    PROCEDURE Div(a,b: INTEGER): INTEGER;
    BEGIN
      RETURN a DIV b
    EXCEPT
      IF IsM2Exception() THEN
        IF M2Exception() = wholeDivException THEN
          IF a < 0 THEN RETURN MIN(INTEGER)
          ELSE          RETURN MAX(INTEGER)
          END;
        END;
      END;
    END Div;
\end{verbatim}

\subsection{Termination}\label{m2:ISO:termination}

During the program termination, finalizations of those static
modules that have started initialization are executed in
reverse order with respect to their initializations (See also
\ref{m2:ISO:final}). The static modules are the program module,
the implementation modules, and any local modules declared in the
module blocks of these modules.

Program termination starts from the first occurrence of the
following event:
\begin{enumerate}
\item end of the program module body is reached
\item a \verb'RETURN' statement is executed in the program module body
\item the standard procedure \verb'HALT' is called
\item an exception was raised and is not handled
\end{enumerate}

The system module {\tt TERMINATION}
\index{Modula-2!TERMINATION}
\index{system modules!TERMINATION}
\index{TERMINATION@{\bf TERMINATION}}
provides facilities for enquiries concerning
the occurrence of termination events.

\verb'    PROCEDURE IsTerminating(): BOOLEAN;'

Returns \verb'TRUE' if any coroutine has inititated
program termination and \verb'FALSE' otherwise.

\verb'    PROCEDURE HasHalted(): BOOLEAN;'

Returns \verb'TRUE' if a call of \verb'HALT'
has been made and \verb'FALSE' otherwise.

\subsection{Coroutines}\label{m2:ISO:coroutines}

The system module {\tt COROUTINES} provides facilities for
coroutines creation, explicit control transfer between
coroutines, and interrupts handling.
\index{system modules!COROUTINES}
\index{Modula-2!COROUTINES}
\index{COROUTINES@{\bf COROUTINES}}
{\bf Note:} Some features can be unavailable in the current release.
See Chapter \ref{limits} for details.

Values of the type {\tt COROUTINE} are created dynamically by a
call of {\tt NEWCOROUTINE} and identify the coroutine in
subsequent operations. A particular coroutine is identified by
the same value of the coroutine type throughout the lifetime of
that coroutine.

\verb'    TYPE COROUTINE;'

The correspondent type was called {\tt PROCESS} in PIM.
From the third edition of PIM, the {\tt ADDRESS} type was used
to identify a coroutine.

\begin{verbatim}
    PROCEDURE NEWCOROUTINE(
                    procBody: PROC;
                    workspace: SYSTEM.ADDRESS;
                    size: CARDINAL;
                    VAR cr: COROUTINE
                    [; initProtection: PROTECTION]);
\end{verbatim}
Creates a new coroutine whose body is given by {\tt procBody}, and
returns the identity of the coroutine in {\tt cr}. {\tt
workspace} is a pointer to the work space allocated to the
coroutine; {\tt size} specifies the size of that workspace in
terms of \verb'SYSTEM.LOC'. {\tt initProtection} is an optional
parameter that specifies the initial protection level of the
coroutine.

An exception is raised (See {\tt coException}) if the value of {\tt
size} is less than the minimum workspace size.

If the optional parameter is omitted, the initial protection of
the coroutine is given by the current protection of the caller.

The created coroutine is initialized in such a way that when
control is first transferred to that coroutine, the procedure
given by {\tt procBody} is called in a normal state. The exception
({\tt coException})  is raised when the {\tt procBody} procedure
attempts to return to its caller. Since the caller has no
exception handler, raising this exception is a termination
event.

The procedure {\tt TRANSFER} can be used to transfer control
from one coroutine to another.

\verb'    PROCEDURE TRANSFER (VAR from: COROUTINE; to: COROUTINE);'

Returns the identity of the calling coroutine in {\tt from} and
transfers control to the coroutine specified by {\tt to}.

\verb'    PROCEDURE CURRENT (): COROUTINE;'

Returns the identity of the calling coroutine.

\subsubsection{Interrupt handling}
\index{interrupt handling}

The {\tt INTERRUPTSOURCE} type is used to identify interrupts.

\verb'    TYPE INTERRUPTSOURCE = INTEGER;'

Programs that use the interrupt handling facilities may be non-portable
since the type is implementation-defined.

\verb'    PROCEDURE ATTACH(source: INTERRUPTSOURCE);'

Associates the specified source of interrupts with the calling
coroutine. More than one source of interrupts may be associated
with a single coroutine.

\verb'    PROCEDURE DETACH(source: INTERRUPTSOURCE);'

Dissociates the specified source of interrupts from the calling
coroutine. The call has no effect if the coroutine is not associated
with source.

\verb'    PROCEDURE IsATTACHED(source: INTERRUPTSOURCE): BOOLEAN;'

Returns TRUE if and only if the specified source of interrupts is
currently associated with a coroutine; otherwise returns FALSE.

\verb'    PROCEDURE HANDLER(source: INTERRUPTSOURCE): COROUTINE;'

Returns the coroutine, if any, that is associated with the source
of  interrupts.  The result is undefined if there is no coroutine
associated with the source.

\verb'    PROCEDURE IOTRANSFER(VAR from: COROUTINE;'\\
\verb'                               to: COROUTINE);'

Returns the identity of the calling coroutine in {\tt from} and
transfers control to the coroutine specified by {\tt to}. On
occurrence of an interrupt, associated with the caller, control is
transferred back to the caller, and {\tt from} returns the identity
of the interrupted coroutine. An exception is raised if the calling
coroutine is not associated with a source of interrupts.

\subsubsection{Protection}\label{m2:ISO:coroutines:prot}

See section \ref{m2:ISO:prot} for information about the type \verb'PROTECTION'.

\verb'    PROCEDURE LISTEN(prot: PROTECTION);'

Momentarily changes protection of the calling coroutine to {\tt prot},
usually lowering it so as to allow an interrupt request to be granted.

\verb'    PROCEDURE PROT(): PROTECTION;'

Returns protection of the calling coroutine.

\subsection{Protection}\label{m2:ISO:prot}

A program module, implementation module or local module may
specify, by including protection in its heading, that
execution of the enclosed statement sequence is protected.

\begin{verbatim}
    ModuleHeading = MODULE ident [ Protection ] ";".
    Protection    = [ ConstExpression ].
\end{verbatim}

A module with protection in its heading is called a directly
protected module. A directly protected procedure is an exported
procedure declared in a protected module.

Protection of a module is provided by surrounding the externally
accessible procedures and module body by calls of access control
procedures. The value of the protection expression is passed to the
call of access control procedures as an actual parameter.

The protection expression should be of the \verb'PROTECTION' type. The
\verb'PROTECTION' type is an elementary type with at least two values:
{\tt INTERRUPTIBLE} and {\tt UNINTERRUPTIBLE}.

Operators \verb+<+, \verb+>+, \verb+<=+ and \verb+>=+ can be used
to compare values of the \verb'PROTECTION' type.
If $x$ is a value of \verb'PROTECTION' type, then $x$ satisfies the conditions:
\begin{center}
        {\tt UNINTERRUPTIBLE} $\leq x\leq$ {\tt INTERRUPTIBLE}
\end{center}

\section{Standard procedures}\label{m2:stdproc}
\index{Modula-2!standard procedures}
\index{standard procedures!Modula-2}

\begin{table}[hbt]
\begin{center}
\begin{tabular}{clp{8cm}}
&\bf Procedure &\bf  Meaning \\ \hline
$\surd$ & \tt ASSERT(x[,n]) & Terminates the program
                 if x$\neq$\verb'TRUE' (See \ref{m2:ext:ASSERT})       \\ \index{ASSERT}
$\surd$ & \tt  COPY(x,v)     & Copies a string: \verb|v := x|       \\ \index{COPY}
&\tt DEC(v[,n])    & \verb|v := v - n|, default n=1             \\ \index{DEC}
&\tt DISPOSE(v)    & Deallocates v\^{} (See \ref{m2:ISO:NEW}) \\ \index{DISPOSE}
&\tt EXCL(v,n)     & \verb|v := v - {n}| \\                    \index{EXCL}
&\tt HALT          & Terminates program execution
                 (See \ref{m2:ext:HALT}) \\                    \index{HALT}
&\tt INC(v[,n])    & \verb|v := v + n|, default n=1 \\         \index{INC}
&\tt INCL(v,n)     & \verb|v := v + {n}| \\                    \index{INCL}
&\tt NEW(v)        & Allocates v\^{} (See \ref{m2:ISO:NEW}) \\
$\surd$ & \tt NEW(v,$x_0$...$x_n$) & Allocates v\^{} of
        length $x_0$...$x_n$  (See \ref{m2:ext:NEW}) \\  % \ldots !!!
\end{tabular}
\end{center}
\caption{Modula-2 proper procedures}\label{table:m2:stdproc}
\end{table}

\begin{table}[htb]
\begin{center}
\begin{tabular}{clp{8cm}}
&\bf Function &\bf  Meaning \\  \hline
&\tt ABS(x)   & Absolute value of $x$ \\              \index{ABS}
$\surd$&\tt ASH(x,n) & Arithmetic shift    \\              \index{ASH}
&\tt CAP(x)   & Corresponding capital letter    \\  \index{CAP}
&\tt CHR(x)   & Character with the ordinal number $x$ \\  \index{CHR}
&\tt CMPLX(x,y) & Complex number with real part $x$
                  and imaginary part $y$           \\ \index{CMPLX}
$\surd$&\tt ENTIER(x) & Largest integer not greater than $x$ \\ \index{ENTIER}
&\tt FLOAT(x) &\tt  VAL(REAL,x) \\                     \index{FLOAT}
&\tt HIGH(v)  &  High bound of the index of $v$  \\    \index{HIGH}
&\tt IM(x)    & Imaginary part of a complex $x$ \\         \index{IM}
&\tt INT(x)   &\tt VAL(INTEGER,x)          \\         \index{INT}
$\surd$&\tt LEN(v[,n]) &  Length of an array
           in the dimension $n$ (default=0) \\    \index{LEN}
&\tt LENGTH(x) & String length           \\         \index{LENGTH}
&\tt LFLOAT(x) &\tt VAL(LONGREAL,x) \\                 \index{LFLOAT}
&\tt MAX(T)   &  Maximum value of type $T$ \\         \index{MAX}
&\tt MIN(T)   &  Minimum value of type $T$ \\         \index{MIN}
&\tt ODD(x)   & $x$ MOD 2 = 1 \\                      \index{ODD}
&\tt ORD(x)   &\tt VAL(CARDINAL,x)    \\               \index{ORD (M2)}
&\tt RE(x)    & Real part of a complex $x$ \\         \index{RE}
&\tt SIZE(T)  & The number of storage units,
               required by a variable of type $T$  \\ \index{SIZE}
&\tt TRUNC(x) & Truncation to the integral part      \\ \index{TRUNC}
&\tt VAL(T,x) & Type conversion            \index{VAL (M2)}
\end{tabular}
\end{center}
\caption{Modula-2 function procedures}\label{table:m2:stdfunc}
\end{table}

This section briefly describes the set of
standard procedures and functions. Some of them are not
defined in the International Standard and are
available only if the option \OERef{M2EXTENSIONS} is set.
The procedure \See{{\tt HALT}}{}{m2:ext:HALT} may have
an additional parameter, if the extensions are enabled .

In the tables (\ref{table:m2:stdproc} and
\ref{table:m2:stdfunc}) of predefined procedures, $v$ stands for a
designator, $x$, $y$ and $n$ --- for expressions, $T$ --- for a
type. Non-standard procedures are marked with $\surd$.

The procedure {\tt COPY} and the functions {\tt ASH}, {\tt ENTIER} and
{\tt LEN} are described in {\em The Oberon-2 Report}.      % !!!

\section{Compatibility}\label{m2:ISO:comp}

This section describes compatibility between entities of
different types. There are three forms of compatibility:
\begin{itemize}
\item expression compatibility (specifying the types that may be
      combined in expressions);
\item assignment compatibility (specifying the type of a value
        that may be assigned to a variable);
\item parameter compatibility (specifying the type of an actual
        parameter that may be passed to a formal parameter).
\end{itemize}
The rules for parameter compatibility are relaxed in the case
where a formal parameter is of a system storage type. This
variation is known as the system parameter compatibility.

In most cases the compatibility rules are the same as described
in PIM. However, we suppose to explicitly list all the
rules.

\subsection{Expression compatibility}\label{m2:ISO:comp:expr}

Two expressions $a$ and $b$ of types $T_a$ and $T_b$
are {\em expression compatible} if any of the following
statement is true:
\renewcommand{\theenumi}{\alph{enumi}}
\begin{enumerate}
\item The types $T_a$ and $T_b$ are identical.
        {\bf Note:} If a type is a subrange type, then only its host type
        matters, therefore values of subranges of the same host
        type are expression compatible with each other and with
        the host type.

\item A type of one expression is a complex type, and the other
      expression is a complex constant.

\item A type of one expression is a real type, and the other
      expression is a real constant.

\item A type of one expression is a whole type, and the other
      expression is a whole constant.

\item A type of one expression is character, and the other
      expression is a string literal of length 0 or 1.
      See also \ref{m2:ISO:strings}.

\begin{verbatim}
    VAR
      char: CHAR;
      ...
      WHILE (char # '') & (char # ".") DO
      ...
\end{verbatim}

\end{enumerate}
\renewcommand{\theenumi}{\arabic{enumi}}

\subsection{Assignment compatibility}\label{m2:ISO:comp:assign}

An expression $e$ of type $T_e$ is {\em assignment compatible}
with the variable $v$ of type $T_v$ if one of the following
conditions holds\footnote{
For an expression of a subrange type only
host type matters.
}:
\renewcommand{\theenumi}{\alph{enumi}}
\begin{enumerate}
\item   $T_v$ is identical to the type $T_e$, and the type is not
        an open array type.

\item   $T_v$ is a subrange of the type $T_e$.

\item   $T_v$ is the CARDINAL type or a subrange of the CARDINAL type
        and $T_e$ is the INTEGER type or $e$ is a whole constant.

\item   $T_v$ is the INTEGER type or a subrange of the INTEGER type
        and $T_e$ is the CARDINAL type or $e$ is a whole constant.

\item   $T_v$ is a real type and $e$ is a real constant.

\item   $T_v$ is a complex type and $e$ is a complex constant.

\item   $T_v$ is a pointer type and $e$ is NIL.

\item   $T_v$ is a procedure type and $e$ is the designator
        of a procedure which has the same structure as
        the procedure type $T_v$ and which has been declared
        at level 0.

\item   $T_v$ is the character type or a subrange of
        the character type and $e$ is a string literal of length 0 or 1.

\item   $T_v$ is an array type having the character type as its
        component type, and $e$ is a string literal of length
        less then or equal to the number of components in arrays of type
        $T_v$\footnote{A string literal is not assignment compatible
        with an array whose component's type is a subrange of
        the character type.}.

\item   $T_v$ is the address type and $T_e$ is a pointer type or
        $T_e$ is the address type and $T_v$ is a pointer type.

\end{enumerate}
\renewcommand{\theenumi}{\arabic{enumi}}

\subsection{Value parameter compatibility}

A formal type is {\em value parameter compatible} with an actual
expression if any of the following statements is true:
\renewcommand{\theenumi}{\alph{enumi}}
\begin{enumerate}
\item   The formal type is constructed from a system storage type
        and is system parameter compatible with the expression.

\item   The formal parameter is an open array, the actual
        parameter is an array type and the component type
        of the formal type is value parameter compatible
        with the component type of the actual type
        \footnote{A formal array parameter with the component's type $T$
        is not parameter compatible with the actual parameter of type $T$.}.

\item   The formal type is assignment compatible with the actual
        parameter.
\end{enumerate}
\renewcommand{\theenumi}{\arabic{enumi}}

\subsection{Variable parameter compatibility}

A formal type is {\em variable parameter compatible} with an actual
variable if any of the following statements is true:
\renewcommand{\theenumi}{\alph{enumi}}
\begin{enumerate}
\item   The formal type is constructed from a system storage type
        and is system parameter compatible with the expression.

\item   The formal parameter is an open array, the actual
        parameter is an array type and the component's type
        of the formal type is variable parameter compatible
        with the component's type of the actual parameter type.

\item   The formal type is identical to the actual parameter type.
\end{enumerate}
\renewcommand{\theenumi}{\arabic{enumi}}

\subsection{System parameter compatibility}\label{m2:ISO:comp:system}

A formal type is {\em system parameter compatible} with an actual
parameter if any of the following statements is true:
\renewcommand{\theenumi}{\alph{enumi}}
\begin{enumerate}
\item   The formal parameter is of the SYSTEM.LOC type
        and the actual
        parameter is of any type $T$ such that {\tt SIZE(T)} is equal to 1.

\item   The formal parameter is of the type

\verb'    ARRAY [0..n-1] OF SYSTEM.LOC'

        and the actual parameter is of any type $T$ such
        that {\tt SIZE(T)} is equal to {\tt n}.

\item   The formal parameter is of the open array type

\verb'    ARRAY OF SYSTEM.LOC'

        and the actual parameter is of any type but not numeric literal.

\item   The formal parameter is of the multi-dimensional open array type

\verb'    ARRAY OF ARRAY [0..n-1] OF SYSTEM.LOC'

        and the actual parameter is of any type $T$ such
        that {\tt SIZE(T)} is a multiple of {\tt n}.

\end{enumerate}
\renewcommand{\theenumi}{\arabic{enumi}}

\iftopspeed\else
%------------------------------------------------------------------------------
\subsection{Function calls in designators}

If the option \OERef{M2EXTENSIONS} is set ON, \xds{} recognizes function 
procedure calls in designators, including  predefined, SYSTEM, and type 
transfer procedures:

\begin{verbatim}
    CurPos(point[i,j])^.x
    GetCurDate().year
    REC_PTR(adr)^.field
\end{verbatim}

%------------------------------------------------------------------------------
\subsection{Open arrray parameter compatibility}

If the option \OERef{M2EXTENSIONS} is set ON, an expression of type \verb'T' 
can be substituted as a parameter whose formal type is \verb'ARRAY OF T'. 

%------------------------------------------------------------------------------
\subsection{Bitwise operations on cardinals}

If the option \OERef{M2EXTENSIONS} is set ON, bitwise operations 
\verb'AND', \verb'OR', and \verb'NOT' are enabled to be applied to cardinals.

%------------------------------------------------------------------------------
\subsection{Logical shift operators}

The '\verb'<<'' and '\verb'>>'' infix operators of logical
left/right shift, defined for cardinal types, are available in
\XDS{} if the option \OERef{M2EXTENSIONS} is set ON.

\fi % \iftopspeed\else


%------------------------------------------------------------------------------
\section{The Modula-2 module SYSTEM}
\label{m2:ISO:system}
\index{Modula-2!SYSTEM}\index{SYSTEM@{\bf SYSTEM}}
\index{system modules!SYSTEM (M2)}

The module {\bf SYSTEM} provides the low-level facilities for
gaining an access to the address and underlying storage of
variables, performing address arithmetic operations and
manipulating the representation of values. Program that use
these facilities may be non-portable.

This module does not exist in the same sense as other libraries
but is hard-coded into the compiler itself. To use the facilities
provided, however, identifiers must be imported in a usual way.

Some of the {\bf SYSTEM} module procedures are generic procedures that
cannot be explicitly declared, i.e. they apply to classes of
operand types or have several possible forms of a parameter list .

The {\bf SYSTEM}  module is the only module specified in the
International  Standard  that  can  be  extended  in  the
implementation. The \XDS{} {\bf SYSTEM} module provides additional
types and procedures.

{\bf Note:} The module {\bf SYSTEM} is different in \ot{}.
See \ref{o2:system} for details.

{\small
\begin{verbatim}
DEFINITION MODULE SYSTEM;

CONST
  BITSPERLOC  = 8;
  LOCSPERWORD = 4;
  LOCSPERBYTE = 1;

TYPE
  LOC;
  ADDRESS = POINTER TO LOC;
  WORD = ARRAY [0 .. LOCSPERWORD-1] OF LOC;
  BYTE = LOC;

PROCEDURE ADDADR(addr: ADDRESS; offset: CARDINAL): ADDRESS;
PROCEDURE SUBADR(addr: ADDRESS; offset: CARDINAL): ADDRESS;
PROCEDURE DIFADR(addr1, addr2: ADDRESS): INTEGER;

PROCEDURE MAKEADR(val: <whole type>): ADDRESS;

PROCEDURE ADR(VAR v: <anytype>): ADDRESS;

PROCEDURE REF(VAR v: <anytype>): POINTER TO <type of the parameter>;

PROCEDURE ROTATE(val: <a packedset type>;
                 num: INTEGER): <type of the first parameter>;

PROCEDURE SHIFT(val: <a packedset type>;
                num: INTEGER): <type of the first parameter>;

PROCEDURE CAST(<targettype>;
               val: <anytype>): <targettype>;

PROCEDURE TSIZE (<type>; ... ): CARDINAL;

(*------------------------------------------------------- *)
(* -------------- non-standard features ----------------- *)

TYPE
  INT8   = <integer 8-bits type>;
  INT16  = <integer 16-bits type>;
  INT32  = <integer 32-bits type>;
  CARD8  = <cardinal 8-bits type>;
  CARD16 = <cardinal 16-bits type>;
  CARD32 = <cardinal 32-bits type>;
  BOOL8  = <boolean 8-bits type>;
  BOOL16 = <boolean 16-bits type>;
  BOOL32 = <boolean 32-bits type>;
  INDEX  = <type of index>
  DIFADR_TYPE = <type that DIFADR function returns>

TYPE (* for use in Oberon *)
  INT  = <Modula-2 INTEGER type>;
  CARD = <Modula-2 CARDINAL type>;

TYPE (* for interfacing to C *)
  int      = <C int type>;
  unsigned = <C unsigned type>;
  size_t   = <C size_t type>;
  void     = <C void type>;

PROCEDURE MOVE(src,dest: ADDRESS; size: CARDINAL);
PROCEDURE FILL(adr : ADDRESS; val : BYTE; size : CARDINAL;);

PROCEDURE GET(adr: ADDRESS; VAR var: SimpleType);
PROCEDURE PUT(adr: ADDRESS; var: SimpleType);

PROCEDURE CC(n: CARDINAL): BOOLEAN;

END SYSTEM.
\end{verbatim}
} % small

\subsection{System types}\label{m2:system:types}
\index{system types}

\begin{description}
\item[LOC] \mbox{}  \index{LOC}

        Values  of the {\tt LOC} type are the uninterpreted contents of
        the   smallest   addressable  unit  of  a  storage  in
        implementation. The value of the call {\tt TSIZE(LOC)} is
        therefore equal to one.

        The  type  {\tt LOC}  was  introduced as a mechanism to resolve
        the problems  with  {\tt  BYTE}  and  {\tt  WORD}  types. Its
        introduction  allows  a consistent handling of both these
        types,  and enables also {\tt WORD}-like types to
        be further introduced, eg:

\verb'    TYPE WORD16 = ARRAY [0..1] OF SYSTEM.LOC;'

        The only operation directly defined for the {\tt LOC} type is
        an assignment. There are special rules affecting parameter
        compatibility for system storage types. See
        \ref{m2:ISO:comp:system} for further details.

\item[BYTE] \mbox{}  \index{BYTE}

        {\tt BYTE}  is defined as {\tt LOC} and has all the properties of the
        type {\tt LOC}.

\item[WORD] \mbox{} \index{WORD}

        The type {\tt WORD} is defined as

\verb'    CONST LOCSPERWORD = 4;'\\
\verb'    TYPE WORD = ARRAY [0..LOCSPERWORD-1] OF LOC;'

        and the value of the call {\tt TSIZE(WORD)} is equal to
        {\tt LOCSPERWORD}.

        The  only  operation  directly defined for the {\tt WORD}
        type is an assignment.  There  are special rules affecting
        parameter compatibility    for    system    storage
        types.   See \ref{m2:ISO:comp:system} for further details.

\item[ADDRESS] \mbox{} \index{ADDRESS}

        The type \verb'ADDRESS' is defined as

\verb'    TYPE ADDRESS = POINTER TO LOC;'

        The \verb'ADDRESS' type is an assignment compatible  with  all
        pointer types and vice versa (See
        \ref{m2:ISO:comp:assign}). A formal variable parameter of
        the \verb'ADDRESS' type is a parameter compatible with an actual
        parameter of any pointer type.

        Variables of type \verb'ADDRESS' are no longer expression
        compatible with \verb'CARDINAL' (as it was in PIM) and they cannot
        directly occur in expressions that include arithmetic
        operators. Functions {\tt ADDADR}, {\tt SUBADR} and {\tt
        DIFADR} were introduced for address arithmetic.

\item[Whole system types] \mbox{}   \index{fixed size types}

        Types {\tt INT8}, {\tt CARD8}, {\tt INT16}, {\tt CARD16},
        {\tt  INT32},  {\tt  CARD32} are guaranteed to contain 8,
        16,  or  32  bits  respectively.

        These types are introduced to simplify constructing the
        interfaces for foreign libraries   (See   Chapter
        \ref{multilang}).  Types  {\tt  SHORTINT}, {\tt LONGINT},
        {\tt  SHORTCARD},  {\tt  LONGCARD}  are  synonyms of {\tt
        INT8},   {\tt   INT32},   {\tt   CARD8},   {\tt   CARD32},
        respectively  (See  also  the  \OERef{M2ADDTYPES} option).
        Types  {\tt  INTEGER}  and {\tt CARDINAL} are synonyms of
        {\tt   INT16}/{\tt  INT32},  {\tt  CARD16}/{\tt  CARD32},
        depending on the target platform. See also the \OERef{M2BASE16} option.

        These types are not described in the International
        Standard.

\item[Boolean system types] \mbox{}

        Types {\tt BOOL8}, {\tt BOOL16}, and {\tt BOOL32} are guaranteed
        to contain 8,16 and 32 bits respectively. By default
        the compiler uses {\tt BOOL8} type for BOOLEAN.
        In some cases (e.g. in the interface to the Windows API) {\tt BOOL16}
        or {\tt BOOL32} should be used instead.

        These types are not described in the International
        Standard.


\item[Bitset system types] \mbox{}

        Types {\tt SET8}, {\tt SET16}, and {\tt SET32} are guaranteed
        to contain 8,16 and 32 bits respectively.
        The predefined type \verb'BITSET' is a synonym for
        \verb'SYSTEM.SET16' or \verb'SYSTEM.SET32', depending
        on the target platform. See also the \OERef{M2BASE16} option.

        These types are not described in the International
        Standard.

\item[Modula-2 whole types] \mbox{}

        Types {\tt INT} and {\tt CARD} are equal to \mt{} {\tt INTEGER}
        and {\tt CARDINAL} types, respectively. These types can
        be used in \ot{} in order to use \mt{} procedures
        in a portable way. See \ref{multilang:m2o2} for further details.

        These types are not described in the International
        Standard.

\item[Interface to C] \mbox{}

        Types  {\tt int},  {\tt unsigned}, \verb|size_t| and {\tt
        void} are introduced to simplify interfacing to C
        libraries. See \ref{multilang:C} for further details.

\end{description}

\subsection{System functions}\label{m2:sysfunc}
\index{Modula-2!system functions}

\index{address arithmetic}

\verb'    PROCEDURE ADDADR(addr: ADDRESS;'\\
\verb'                     offs: CARDINAL): ADDRESS;'

\index{SYSTEM!ADDADR}
Returns an address given by {\tt (addr + offs)}.
The subsequent use of the calculated address may raise an exception.

\verb'    PROCEDURE SUBADR(addr: ADDRESS;'\\
\verb'                     offs: CARDINAL): ADDRESS;'

\index{SYSTEM!SUBADR}
Returns an address given by {\tt (addr - offs)}.
The subsequent use of the calculated address may raise an exception.

\ifgenc
The result of functions {\tt ADDADR} and {\tt SUBADR} is meaningful
if {\tt addr} points into some object and the calculated address
points into the same object or points to the next byte after
the object.
\fi

\verb'    PROCEDURE DIFADR(addr1,addr2: ADDRESS): INTEGER;'

\index{SYSTEM!DIFADR}
Returns the difference between addresses {\tt (addr1 - addr2)}.

\ifgenc
The result of this function is meaningful if
\verb|addr1| and \verb|addr2| point into the same
object or to the next byte after the object.
\fi

\verb'    PROCEDURE MAKEADR(val: <whole type>): ADDRESS;'

The function is used to construct a value of the ADDRESS type
from the value of a whole type.

{\bf Note:} The International Standard
does not define the number and types of the parameters.
Programs that use this procedure may be non-portable.

\verb'    PROCEDURE ADR(VAR v: <any type>): ADDRESS;'

\index{SYSTEM!ADR (M2)}
Returns the address of the variable {\tt v}.

\verb'    PROCEDURE CAST(<type>; x: <any type>): <type>;'

\index{SYSTEM!CAST}
The function {\tt CAST} can be used (as a type transfer function)
to interpret a value of any type other than a numeric literal
value as a value of another type
\footnote{The International Standard forbids the use of
the PIM style type transfer, like {\tt CARDINAL(x)}.}.

The  value  of  the  call  {\tt  CAST(Type,val)}  is an unchecked
conversion of {\tt val} to the type {\tt Type}.
If {\tt SIZE(val) = TSIZE(Type)},
the  bit pattern representation of the result is
the  same  as  the  bit  pattern  representation  of  {\tt  val};
otherwise the result and the value of {\tt val} have the same bit
pattern  representation  for  a  size equal to the smaller of the
numbers of storage units.

The given implementation may forbid some combinations  of
parameter types.

{\bf Note:} In \ot{} module SYSTEM, the respective procedure
is called \verb'VAL'.

\verb'    PROCEDURE TSIZE(Type; ... ): CARDINAL;'

\index{SYSTEM!TSIZE}
Returns the number of storage units (\verb'LOC's) used to
store the value of the specified type. The extra parameters,
if present, are used to distinguish variants in a variant
record and must be constant expressions\footnote{Those constant
expressions are ignored in the current release.}.

\Example
\begin{verbatim}
    TYPE
      R = RECORD
        CASE i: INTEGER OF
          |1: r: REAL;
          |2: b: BOOLEAN;
        END;
      END;

      ... TSIZE(R,1) ...

\end{verbatim}

The value of {\tt TSIZE(T)} is equal to {\tt SIZE(T)}.

\subsubsection{Packedset functions}

Values of packedset types are represented as sequences of bits\footnote{The current
implementation does not distinguish between set and packedset types.}.
The bit number 0 is the least significant bit
for a given platform. The following is true, where {\tt v} is a variable
of the type {\tt CARDINAL}:
\begin{verbatim}
    CAST(CARDINAL,BITSET{0}) = VAL(CARDINAL,1)
    SHIFT(CAST(BITSET,v),1)  = v * 2
    SHIFT(CAST(BITSET,v),-1) = v DIV 2
\end{verbatim}

{\bf Note:} The functions {\tt ROTATE} and {\tt SHIFT} can be applied
to a set with size less than or equal to the size of {\tt BITSET}.

\verb'    PROCEDURE ROTATE(x: T; n: integer): T;'

\index{SYSTEM!ROTATE}
Returns the value of \verb|x| rotated \verb|n| bits to the left (for
positive \verb|n|) or to the right (for negative \verb|n|).

\verb'    PROCEDURE SHIFT(x: T; n: integer): T;'

\index{SYSTEM!SHIFT}
Returns the value of \verb|x| logically shifted \verb|n| bits to
the left (for positive \verb|n|) or to the right (for negative
\verb|n|).

{\bf Warning:} The result of \verb'SHIFT(x,n)' , where \verb'n' is greater 
than the number of elements in \verb'T', is undefined.

\subsubsection{Non-standard functions}
\label{m2:sysfunc:nonstandard}

\verb'    PROCEDURE CC(n: whole constant): BOOLEAN;'

\index{SYSTEM!CC}

\ifcomment
Returns TRUE if the corresponding condition flag is set ($0\leq n\leq 15$).
See table \ref{table:cc} for further information.
\begin{center}
\begin{table}
\begin{flushleft}
\begin{tabular}{|r|l|}\hline
\bf Condition    &\bf Meaning \\ \hline
 0 &   overflow flag                    \\
 1 &   not overflow flag                \\
 2 &   carry flag                       \\
 3 &   unsigned greater or equal        \\
 4 &   equal                            \\
 5 &   not equal                        \\
 6 &   unsigned less or equal           \\
 7 &   unsigned greater                 \\
 8 &   sign flag                        \\
 9 &   not sign flag                    \\
10 &   parity flag                      \\
11 &   not parity flag                  \\
12 &   signed less                      \\
13 &   signed greater or equal          \\
14 &   signed less or equal             \\
15 &   signed greater                   \\
\hline
\end{tabular}
\end{flushleft}
\caption{Condition flags}\label{table:cc}
\end{table}
\end{center}
\else
Returns TRUE if the corresponding condition flag is set.
The function is not implemented in the current release.
\fi

\verb'    PROCEDURE REF(VAR v: <anytype>):'\\
\verb'                  POINTER TO <type of the parameter>;'

\index{SYSTEM!REF (M2)}

Returns the pointer to the variable {\tt v}. See also
\ref{multilang:parmcomp}.

\verb'    PROCEDURE BIT(adr: T; bit: INTEGER): BOOLEAN;'

\index{SYSTEM!BIT}
Returns bit \verb|n| of \verb|Mem[adr]|.
\verb|T| is either ADDRESS or whole type.

\subsection{System procedures}
\index{Modula-2!system procedures}
\label{m2:system_procs}

{\bf Note:} all these procedures are non-standard.

\verb'    PROCEDURE MOVE (src, dst: ADDRESS; size: CARDINAL);'

\index{SYSTEM!MOVE}
Copies \verb|size| bytes from the memory location specified
by \verb|src| to the memory location specified by \verb|dst|.

{\bf Warning:} No check for area overlap is performed. The behaviour
of \verb'SYSTEM.MOVE' in case of overlapping areas is undefined.

\verb'    PROCEDURE FILL(adr : ADDRESS; val : BYTE; size : CARDINAL;);'

\index{SYSTEM!FILL}
Fills the memory block of size \verb'size' starting from the memory
location specified by \verb'adr' with the value of \verb'val'
\ifgencode
using a very efficient code.
\else
using the \verb'memset' C library function.
\fi

\verb'    PROCEDURE GET (adr: ADDRESS; VAR v: SimpleType);'\\
\verb'    PROCEDURE PUT (adr: ADDRESS;     x: SimpleType);'

\index{SYSTEM!GET}\index{SYSTEM!PUT}
Gets/puts a value from/to address specified by \verb|adr|.
The second parameter cannot be of a record or array type.

\begin{verbatim}
    VAR i: INTEGER;

      GET (128, i);   (* get system cell value *)
      i := i+20;      (* change it             *)
      PUT (128, i);   (* and put back          *)
\end{verbatim}

\verb'    PROCEDURE CODE(...);'

\index{SYSTEM!CODE}
\ifgencode
The procedure is intended to embed a sequence of machine
instructions  directly into the generated code. The procedure is not
implemented in the current release.
\fi
\ifgenc
The CODE procedure accepts a sequence of strings as its
parameter. This procedure inserts a string which is the concatenation
of the parameters into the generated C code.

\Example
\begin{verbatim}
    PROCEDURE disable;      (* disable interrupts *)
    BEGIN
      SYSTEM.CODE ("asm di;");
    END disable;
\end{verbatim}
will produce

\verb'    asm di;'

\fi

%----------------------------------------------------------------

\section{Language extensions}\label{m2:ext}

{\bf Warning:} Using extensions may cause problems with software
portability to other compilers.

In the standard mode the \xds{} \mt{} compiler is ISO compliant
(See \ref{m2:ISO}). A set of language extensions may be enabled
using the \OERef{M2EXTENSIONS} and \OERef{M2ADDTYPES} options.

The main purposes of supporting the language extensions are:
\begin{itemize}
\item to improve interfacing with other languages
      (See Chapter \ref{multilang})
\item to simplify migration from \mt{} to \ot{}
\item to implement some useful features not found in ISO \mt{}
\item to provide backward compatibility with previous releases
\end{itemize}

\subsection{Lexical extensions}
\index{Modula-2!lexical extensions}

\subsubsection{Comments}

\mextonly

As well as \verb|(**)|, there is another valid format for comments in
the source texts. The portion of a line from ``\verb|--|''
to the end is considered as a comment.

\begin{verbatim}
      VAR i: INTEGER; -- this is a comment
    --(*
      i:=0; (* this line will be compiled *)
    --*)
\end{verbatim}

\subsubsection{Numeric constants}

\mextonly

Both \mt{} and \ot{}  syntax rules for the numeric and character
representations may be used.

\begin{verbatim}
    Number      = [ "+" | "-" ] Integer | Real.
    Integer     = digit { digit }
                | octalDigit { octalDigit } "B"
                | digit { hexDigit } "X".
    Real        = digit { digit } "." { digit } [ ScaleFactor ].
    ScaleFactor = ( "E" | "D" ) [ "+" | "-" ] digit {digit}.

    Character   = '"' char '"' | "'" char "'"
                | digit {hexDigit} "H"
                | octalDigit {octalDigit} "C".
\end{verbatim}

\Examples
\begin{verbatim}
    1991        1991 (decimal)
    0DH         13   (decimal)
    15B         13   (decimal)
    41X         "A"
    101C        "A"
\end{verbatim}

{\bf Note:} the symbol "\verb'D'" in a \verb'ScaleFactor'
denotes a \verb'LONGREAL' value.

\subsection{Additional numeric types}\label{m2:ext:numeric}
\index{Modula-2!numeric types}

\maddonly

The compiler option \OERef{M2ADDTYPES} introduces
the following additional numeric types:

\ifonline
\begin{flushleft}
\begin{tabular}{lll}
1. &\bf SHORTINT  & integers between $-128$ and $127$           \\
2. &\bf LONGINT   & integers between $-2**31$ and $2**31-1$     \\
3. &\bf SHORTCARD & unsigned integers between $0$ and $255$     \\
4. &\bf LONGCARD  & unsigned integers between $0$ and $2**32-1$ \\
\end{tabular}
\end{flushleft}
\else
\begin{flushleft}
\begin{tabular}{lll}
1. &\bf SHORTINT  & integers between $-128$ and $127$            \\
2. &\bf LONGINT   & integers between $-2^{31}$ and $2^{31}-1$    \\
3. &\bf SHORTCARD & unsigned integers between $0$ and $255$      \\
4. &\bf LONGCARD  & unsigned integers between $0$ and $2^{32}-1$ \\
\end{tabular}
\end{flushleft}
\fi

The following terms for groups of types will be used:
\begin{flushleft}
{\em Real  types\/} for (\verb'REAL', \verb'LONGREAL')                       \\
{\em Integer  types\/} for (\verb'SHORTINT', \verb'INTEGER', \verb'LONGINT')        \\
{\em Cardinal types\/} for (\verb'SHORTCARD', \verb'CARDINAL', \verb'LONGCARD')     \\
{\em Whole    types\/} for {\em integer\/} and {\em cardinal types} \\
{\em Numeric  types\/} for {\em whole\/} and {\em real types}          \\
\end{flushleft}

All integer types are implemented as subranges of
internal compiler integer types. Therefore, according to the
compatibility rules (See \ref{m2:ISO:comp}), the values of different
integer types can be mixed in the expressions. The same holds for
cardinal  types. A mixture of integer and cardinal types
is not allowed in expressions. As in \ot{}, the numeric types form a
hierarchy, and larger types include (i.e. can accept the values
of) smaller types:
$$
\mbox{{\tt LONGREAL}} \subseteq \mbox{{\tt REAL}} \subset \mbox{\em whole types}
$$

Type compatibility in expressions is extended according to the
following rules (See \ref{m2:ISO:comp:expr}):
\begin{itemize}
\item   The  type of the result of an arithmetic or relation
        operation is the smallest type which includes the types
        of both operands.

\item   Before the operation, the values of  both  operands  are
        converted to the result's type.
\end{itemize}

For instance, if the following variables are defined:
\begin{verbatim}
     s: SHORTCARD;
     c: CARDINAL;
     i: INTEGER;
     l: LONGINT;
     r: REAL;
     lr: LONGREAL;
\end{verbatim}
then
\begin{center}
\begin{tabular}{lll}
\bf Expression &\bf Meaning &\bf Result type \\ \hline
\tt s + c        &\tt VAL(CARDINAL,s) + c       & CARDINAL \\
\tt l * i        &\tt l * VAL(LONGINT,i)        & LONGINT \\
\tt r + 1        &\tt r + VAL(REAL,1)           & REAL    \\
\tt r = s        &\tt r = VAL(REAL,s)           & BOOLEAN \\
\tt r + lr       &\tt VAL(LONGREAL,r) + lr      & LONGREAL \\
\tt c + i        &\tt not allowed               &           \\
\end{tabular}
\end{center}

The assignment compatibility rules  are  also  extended  (See
\ref{m2:ISO:comp:assign}),  so an expression $e$ of type $T_e$ is
assignment  compatible with a variable $v$ of type $T_v$ if $T_e$
and $T_v$ are of numeric types and $T_v$ includes $T_e$. Cardinal
types  and integer types are assignment compatible. The compiler
generates the range checks whenever necessary.

\paragraph{Examples (see declarations above):}
\begin{center}
\begin{tabular}{ll}
\bf Statement &\bf Comment \\ \hline
\tt i:=c;     & {\tt INTEGER} and {\tt CARDINAL} are assignment compatible \\
\tt i:=s;     & {\tt INTEGER} and {\tt SHORTCARD} are assignment compatible \\
\tt l:=i;     & {\tt LONGINT} and {\tt INTEGER} are subranges of the same
                host type                                                  \\
\tt r:=i;     & {\tt REAL}$\subset$ {\tt INTEGER}                           \\
\tt r:=c;     & {\tt REAL}$\subset$ {\tt CARDINAL}                          \\
\tt lr:=r;    & {\tt LONGREAL}$\subseteq$ {\tt REAL}                          \\
\end{tabular}
\end{center}

\subsection{Type casting}

\mextonly

In ISO Modula-2, the second parameter of the \verb'SYSTEM.CAST' procedure
can not be a numeric literal. \XDS{} provides numeric literal casting as
an extension:

\begin{verbatim}
    VAR
      c: CARDINAL;
    BEGIN
      (* Ok if M2EXTENSIONS is ON *)
      c := SYSTEM.CAST(CARDINAL,-1); 
\end{verbatim}

\subsection{Assignment compatibility with BYTE}

\mextonly

An expression of type {\tt CHAR}, {\tt BOOLEAN}, {\tt SHORTCARD}, {\tt SHORTINT},
{\tt SYSTEM.INT8}, or {\tt SYSTEM.CARD8} can
be  assigned to a variable of type  {\tt BYTE} or passed as an actual
parameters to a formal parameter of type {\tt BYTE}.

\subsection{Dynamic arrays}\label{m2:ext:dynarrs}
\index{Modula-2!dynamic arrays}

\mextonly

\xds{} allows \ot{} style dynamic arrays to be used
according to the \ot{} rules.

An open array is an array type with no lower and upper bound
specified, i.e. ARRAY OF SomeType. Open arrays may be used only
in procedure parameter lists or as a pointer base type.

\verb'    TYPE String = POINTER TO ARRAY OF CHAR;'

Neither variables nor record fields may be
of open array type.

If the designator type is formally an open array, then the only
operations allowed with it are indexing and passing it to a
procedure.

The extended versions of standard procedures {\tt NEW} and
{\tt DISPOSE} can be used to create and delete the dynamic arrays
(See \ref{m2:ext:NEW}).

\Example
\begin{verbatim}
    TYPE
      VECTOR = ARRAY OF REAL;
        (* 1-dim open array *)
      Vector = POINTER TO VECTOR;
        (* pointer to open array *)
      MATRIX = ARRAY OF VECTOR;
        (* 2-dim open array *)
      Matrix = POINTER TO MATRIX;
        (* pointer to this *)

    VAR
       v: Vector;
       m: Matrix;

    PROCEDURE ClearVector(VAR v: VECTOR);
      VAR i: CARDINAL;
    BEGIN
      FOR i := 0 TO HIGH (v) DO v[i] := 0 END;
    END ClearVector;

    PROCEDURE ClearMatrix(VAR m: Matrix);
      VAR i: CARDINAL;
    BEGIN
      FOR i := 0 TO HIGH (m) DO ClearVector(m[i]) END;
    END ClearMatrix;

    PROCEDURE Test;
    BEGIN
      NEW(v, 10);
      NEW(m, 10, 20);
      ClearVector(v^);
      ClearMatrix(m^);
      v^[0] := 1;
      m^[1][1] := 2;
      m^[2,2] := 1000;
      DISPOSE(v);
      DISPOSE(m);
    END Test;
\end{verbatim}

\subsection{Constant array constructors}\label{m2:constarrays}
\index{Modula-2!array constructors}
\mextonly

\xds{} allows the declaration of constant arrays in the form

\verb'    ARRAY OF QualIdent "{" ExprList "}"'

{\tt QualIdent} should refer to a basic type, range or enumeration  type,  and
all expressions within {\tt ExprList} should be of that type.

{\bf Note:} structured types and non-constant expressions
are not allowed.

The actual type of such a constant is \verb|ARRAY [0..n] OF QualIdent|,
where \verb|n+1| is the number
of expressions in \verb'ExprList'.

\verb'    CONST table = ARRAY OF INTEGER {1, 2+3, 3};'

Constant arrays are subject to the same rules as all other constants,
and may be read as a normal array.

In some cases constructors of this form are more convenient than
ISO standard value constructors (See \ref{m2:ISO:aggregates}),
because you do not need to declare a type and to calculate
manually the number of expressions. However, to make your
programs more portable, we recommend to use the standard features.

\subsection{Set complement}\index{Modula-2!set complement}

\mextonly

As in \ot{}, an unary minus applied to a set denotes the complement
of that set, i.e. $-x$ is the set of all values which are not the elements
of $x$.

\begin{verbatim}
    TYPE SmallSet = SET OF [0..5];
    VAR x, y: SmallSet;
    BEGIN
      x := SmallSet{1,3,5};
      y := -x;                  (* y = {0, 2, 4} *)
      y := SmallSet{0..5} - x;  (* y = {0, 2, 4} *)
    END;
\end{verbatim}


%----------------------------------------------------------------
\subsection{Volatile declarations}\label{m2:ext:volatile}
\index{volatile declarations}
\index{Modula-2!volatile declarations}

\mextonly

The compiler option \OERef{SYSTEMFLAGEXT} enables volatile 
declarations via extension of the \TRef{direct language specification}{multilang:direct} 
syntax. It provides a flexible way to declare volatile variables, 
fields, types and procedure VAR parameters as well.

The volatile specification clause can be specified as \verb|["Volatile"]|
immediately following the keyword "{\tt POINTER}" or "{\tt TO}" and can be
placed after a name of a variable, field, type and procedure VAR parameter.
A language string and "{\tt Volatile}" specification can appear together 
in one clause,  separated by the a comma.   

\Example
\begin{verbatim}
  TYPE
    MyType1 ["C", Volatile"] = POINTER TO INTEGER;

  VAR
    v1: MyType1;
    v2 ["Volatile"]: POINTER TO INTEGER;
    v3: POINTER ["Volatile"] TO INTEGER;
    
  PROCEDURE myProc(VAR a ["Volatile"]: MyRecord);
\end{verbatim}

In this example all variables: "{\tt v1}", "{\tt v2}", "{\tt v3}" 
are declared as volatile pointer to a integer value. To declare a 
(non-volatile) pointer to a volatile object the volatile specification 
should be placed after keyword "{\tt TO}".

\Example
\begin{verbatim}
  TYPE
    MyRecord = RECORD
      field1: POINTER TO ["Volatile"] Device;
      field2: POINTER ["Volatile"] TO Device;
      field3: POINTER ["Volatile"] TO ["Volatile"] Device;
    END;
\end{verbatim}

In the above example, "{\tt field1}" is not itself volatile but what 
it points to is. Whereas "{\tt field2}" is volatile but what it points 
to isn't and "{\tt field3}" is volatile as well as what it points to.

The absolute address of a variable, direct language specification and 
volatile specification can be placed into the same square bracket syntax.
The absolute address should be specified first. Empty brackets are not 
required anymore before direct language specification and volatile 
specification in the variable declaration.

\Example
\begin{verbatim}
  VAR
    Store4 [CAST(ADDRESS, Base+300H), "C", "Volatile"]: MyRecord;
\end{verbatim}

See also the \ref{m2:ext:volatile:param} option.


%----------------------------------------------------------------
\subsection{Volatile parameters}\label{m2:ext:volatile:param}
\index{volatile parameters}
\index{Modula-2!volatile parameters}

\mextonly

If the option \OERef{SYSTEMFLAGEXT} is set ON, then in a formal 
parameter section, the volatile specification \verb|["Volatile"]|
can be placed after the name of a VAR parameter. Such a parameter
is called volatile; its value may change in a way that can not be
determined at compile time. As a result, the optimizer will not
eliminate any operation involving that parameter. 

We recommend to use volatile parameters to accept volatile variables 
as actual parameters.

\Example
\begin{verbatim}
  PROCEDURE MyProcedure (VAR dest ["Volatile"]: INTEGER);
  BEGIN
    ...
  END MyProcedure;
  
  VAR
    common ["Volatile"]: CARDINAL;
    
  BEGIN
    MyProcedure(common)
  END ...
\end{verbatim}

The volatile specification \verb|["Volatile"]| can also be placed
immediately following the keyword "{\tt VAR}" in the VAR parameters
declaration. In this case all VAR parameters in the given clause
are volatile.

\Example
\begin{verbatim}
  PROCEDURE MyProcedure2 (VAR ["Volatile"] Param1, Param2 : CARD32);
\end{verbatim}

To declare a procedure type that takes volatile parameters, the volatile 
specification \verb|["Volatile"]| can be placed immediately following 
the keyword "{\tt VAR}" parameters' type specification.

\Example
\begin{verbatim}
  TYPE
    MyProcedureType = PROCEDURE (VAR ["Volatile"] CARD32);
\end{verbatim}

See also \ref{m2:ext:volatile}.


If the option \OERef{SYSTEMFLAGEXT} is set OFF, then in a formal parameter 
section, the symbol \verb|"+"| may be placed after the name of a VAR parameter 
to declare that parameter as volatile.

\Example
\begin{verbatim}
  PROCEDURE OldStileProcedure(VAR dest+: INTEGER);
\end{verbatim}


%----------------------------------------------------------------
\subsection{Read-only parameters}\label{m2:ext:RO_param}
\index{read-only parameters}
\index{Modula-2!read-only parameters}

\mextonly

In a formal parameter section, the symbol \verb|"-"| may be
placed after the name of a value parameter.  Such a parameter is
called {\em read-only}; its value can not be changed in the
procedure body. Read-only parameters do not need to be copied before
procedure activation; this enables procedures
with structured parameters to be more effective.

For \verb'ARRAY' and \verb'RECORD' read-only parameters,
the array elements and record fields are protected.
Read-only parameters cannot be used in definition modules.

We recommend to use read-only parameters with care. The compiler
does not check that the read-only parameter is not modified via
another parameter or a global variable.

\Example
\begin{verbatim}
    PROCEDURE Foo(VAR dest: ARRAY OF CHAR;
                   source-: ARRAY OF CHAR);
    BEGIN
      dest[0]:='a';
      dest[1]:=source[0];
    END Foo;
\end{verbatim}

The call \verb|Foo(x,x)| would produce a wrong result, because
the first \verb'Foo' statement changes the value of \verb|source[0]|
(\verb|source| is not copied and points to the same location
as \verb|dest|).



\subsection{Variable number of parameters}\label{m2:SEQ:param}
\index{SEQ parameters}
\index{Modula-2!SEQ parameters}

\mextonly

The last formal parameter of a procedure may be declared as a
``sequence of bytes'' (SEQ-parameter). In a procedure call, any
(possibly empty) sequence of actual parameters of any types
may be substituted in place of that parameter.
Only the declaration

\verb'    SEQ name: SYSTEM.BYTE'

is allowed. A procedure may have only one SEQ parameter, and it must be
the last element of the formal parameters list.

Within the procedure, sequence parameters are very similar to open
array parameters. This means that :
\begin{itemize}
\item the {\tt HIGH} function can be applied to the parameter;
\item a {\tt SEQ} actual parameter may be subsequently passed to another procedure
\item the \verb|i|-th byte of the sequence \verb|s|
      can be accessed as \verb|s[i]|, like an array element.
\end{itemize}

An array of bytes, which is passed to a procedure as a formal
SEQ-parameter, is formed as follows:
\begin{itemize}
\item
  values of all actual parameters forming the sequence are represented
  as described below and concatenated into an array in their textual order
\item integer values are converted to {\tt LONGINT}
\item {\tt BOOLEAN}, {\tt CHAR}, cardinal and enumeration values are converted
      to {\tt LONGCARD}
\item values of range types are converted according to their
      base types
\item real values are converted to {\tt LONGREAL}
\item values of pointer, opaque and procedure types
are converted to {\tt ADDRESS}
\item
    a structured value (record or array) is interpreted
    as an array of bytes and passed as a sequence
    of:
    \begin{itemize}
    \item the address of the structure
    \item a zero 32-bit word (reserved for future extensions)
    \item size of the structure (in LOCs) minus one
    \end{itemize}
\end{itemize}

\ifgencode
  See \ref{lowlevel:seqrep} for further information. 
\fi
\ifgenc
  See \ref{maptoc:procdcl:seqrep} for further information. 
\fi


\subsection{Read-only export}\label{RO:export}
\index{Modula-2!read-only export}

\mextonly

The \ot{} read-only export symbol "-", being specified after a variable or field
identifier in a definition module will define the identifier as read-only
for any client.
Only the module in which a read-only variable or field is declared may
change its value.

The compiler will not allow the value
of a read-only exported object to be changed explicitly (by
an assignment) or implicitly (by passing it as a VAR parameter).

For read-only variables of an array or record type, both array elements
and record fields are also read-only.

\paragraph{Example (an excerpt from a definition module):}
\begin{verbatim}
    TYPE Rec = RECORD
      n-: INTEGER;
      m : INTEGER;
    END;

    VAR
      in-: FILE;
      x-: Rec;
\end{verbatim}

\subsection{Renaming of imported modules}
\index{Modula-2!renaming in import clause}
\mextonly

An imported module can be renamed inside the importing module. The real
name of the module becomes invisible.
\begin{verbatim}
    Import = IMPORT [ Ident ":=" ] Ident
              { "," [ Ident ":=" ] ident } ";".
\end{verbatim}

\Example
\begin{verbatim}
    MODULE test;
    IMPORT vw := VirtualWorkstation;

    VAR ws: vw.Station;

    BEGIN
      ws := vw.open();
    END test.
\end{verbatim}

\subsection{NEW and DISPOSE for dynamic arrays}\label{m2:ext:NEW}
\index{Modula-2!NEW and DISPOSE}
\index{NEW (M2)}\index{DISPOSE}
\index{memory management}
\index{DYNALLOCATE}
\index{DYNDEALLOCATE}

Standard procedures {\tt NEW} and {\tt DISPOSE} can be applied to
variables of a dynamic array type (See \ref{m2:ext:dynarrs}).
Procedures {\tt DYNALLOCATE} and {\tt DYNDEALLOCATE} have to be visible
in the calling context. Their headers and semantics are described below.

\begin{verbatim}
    PROCEDURE DYNALLOCATE(VAR a: ADDRESS;
                           size: CARDINAL;
                            len: ARRAY OF CARDINAL);
\end{verbatim}

The procedure must allocate a dynamic array and return its address in \verb'a'.
\verb|size| is the size of the array base type (the size of an element)
and \verb|len[i]| is the length of the array in $i$-th dimension.

\begin{verbatim}
    PROCEDURE DYNDEALLOCATE(VAR a: ADDRESS;
                         size,dim: CARDINAL);
\end{verbatim}

The procedure must deallocate a dynamic array,
where \verb|size| is the size of an element and \verb|dim|
is the number of dimensions.

{\bf Note:} In most cases, default implementation of these procedures
may be used. The \OERef{STORAGE} option controls whether the default
memory management should be enabled.

A dynamic array is represented as a pointer to a so-called {\it array
descriptor}
\ifgenc (See \ref{maptoc:dynarr}). \fi
\ifgencode (See \ref{lowlevel:opendesc}). \fi

\subsection{HALT}\label{m2:ext:HALT}
\index{HALT}
\mextonly

An optional integer parameter is allowed for the \verb'HALT' procedure.

\verb'    PROCEDURE HALT ([code: INTEGER]);'

\verb|HALT| terminates the program execution with an optional return code.
Consult your operating system/environment documentation for more details.

\subsection{ASSERT}\label{m2:ext:ASSERT}\index{ASSERT}
\index{ASSERT}
\mextonly

The procedure \verb'ASSERT' checks its boolean parameter and terminates
the program if it is not \verb'TRUE'. The second optional parameter
denotes {\em task termination code}. If it is omitted, a standard value
is assumed.

\verb'PROCEDURE ASSERT(cond: BOOLEAN [; code: INTEGER]);'

A call \verb|ASSERT(expr,code)| is equivalent to

\verb'IF NOT expr THEN HALT(code) END;'

\section{Source code directives}\label{m2:pragmas}

Source code directives (or pragmas) are used to set compilation
options in the source text and to select specific pieces of the source
text to be compiled (conditional compilation).
The ISO \mt{} standard does not describe pragma syntax.
\XDS{} supports source code directives
in both \mt{} and \ot{}. The syntax described in
{\em The Oakwood Guidelines for the \ot{} Compiler Developers}
is used.

\subsection{Inline options and equations}\label{m2:pragmas:inline}
\index{inline options}
\index{inline equations}

In some cases it is more desirable to set a compiler option or
equation within the source  text. Some compiler options,  such as
\OERef{MAIN}, are more meaningful in the source file before the
module header, and some, such as run-time checks, even between statements.

\xds{} allows options to be changed in the source text by using standard
ISO  pseudo comments \verb|<* ... *>|\footnote{The old pragma style
{\tt (*\$..*)} is supported to provide backward compatibility,
but the compiler reports the ``obsolete syntax'' warning.}
Some options can only be placed in the source text before the module
header (i.e. before keywords \verb'IMPLEMENTATION', \verb'DEFINITION',
and \verb'MODULE'). These options will be ignored if found elsewhere
in the source text. See \ref{opt:bool:list} for more details.

The format of an inline option or equation setting is described by
the following syntax:
\begin{verbatim}
Pragma     = "<*" PragmaBody "*>"
PragmaBody = PUSH | POP | NewStyle | OldStyle
NewStyle   = [ NEW ] name [ "+" | "-" | "=" string ]
OldStyle   = ("+" | "-") name
\end{verbatim}
\index{<* *>@\verb'<* *>'}

{\tt NewStyle} is proposed as the Oakwood standard for \ot{},
{\tt OldStyle} is the style used in the previous \xds{} releases.  All
option names are case-independent.  If {\tt OldStyle} is used, 
there should be  no  space between \verb|<*| and \verb|+| or \verb|-|
{\tt OldStyle} does not allow to declare a new option or equation
and to change an equation value.

In all cases, the symbol \verb|+| sets the corresponding
option ON, and the symbol \verb|-| sets it OFF.

{\tt PUSH} and {\tt POP} keywords may be used to save and restore
the whole state of options and equations.

\Examples
\begin{verbatim}
    PROCEDURE Length(VAR a: ARRAY OF CHAR): CARDINAL;
      VAR i: CARDINAL;
    BEGIN
      <* PUSH *>              (* save state *)
      <* CHECKINDEX - *>      (* turn CHECKINDEX off *)
      i := 0;
      WHILE (i<=HIGH(a)) & (a[i]#0C) DO INC(i) END;
      <* POP *>               (* restore state *)
      RETURN i;
    END Length;
\end{verbatim}
\begin{verbatim}
    <* ALIGNMENT = "2" *>
    TYPE
      R = RECORD      (* This record is 6 bytes long *)
            f1: CHAR;
            f2: CARDINAL;
          END;
\end{verbatim}

\subsection{Conditional compilation}\label{m2:pragmas:cc}
\index{conditional compilation}

It is possible to use conditional compilation with \mt{} and
\ot{}\footnote{only if the {\bf O2ISOPRAGMA} option is set ON}
compilers via the standard ISO pragma notation \verb|<* *>|.
Conditional compilation statements can be placed anywhere in the source
code. The syntax of the conditional compilation \verb'IF' statement
follows:
\begin{verbatim}
    IfStatement      = <* IF Expression THEN *> text
                     { <* ELSIF Expression THEN *> text }
                     [ <* ELSE *> text ]
                       <* END *>
    Expression       = SimpleExpression
                       [ ("=" | "#") SimpleExpression].
    SimpleExpression = Term { "OR" Term}.
    Term             = Factor { "&" Factor}.
    Factor           = Ident | string |
                       "DEFINED" "(" Ident ")" |
                       "(" Expression ")" |
                       "~" Factor | "NOT" Factor.
    Ident            = option | equation.
\end{verbatim}

An operand in an expression is either a name of an option or equation or
a string literal. An option has the string value \verb'"TRUE"', if it is
currently set ON and \verb'"FALSE"', if it is currently set off
or was not defined at all. The compiler will report a warning if an
undeclared option or equation is used as a conditional compilation identifier.

The comparison operators "\verb'='" and "\verb'#'" are not case sensitive.

See also the section \ref{opt:COMPILER}.

\Examples
\begin{verbatim}
    IMPORT lib :=
          <* IF  __GEN_X86__ THEN *> MyX86Lib;
          <* ELSIF __GEN_C__ THEN *> MyCLib;
          <* ELSE *> *** Unknown ***
          <* END *>
\end{verbatim}

\begin{verbatim}
    CONST Win = <* IF Windows THEN *> TRUE
                <* ELSE *> FALSE
                <* END *>;
\end{verbatim}

\begin{verbatim}
    <* IF DEFINED(Debug) & (DebugLevel = "2") THEN *>
       PrintDebugInformation;
    <* END *>;
\end{verbatim}

\begin{verbatim}
    <* IF target_os = "OS2" THEN *>
      Strings.Capitalize(filename);
      <* IF NOT HPFS THEN *>
        TruncateFileName(filename);
      <* END *>
    <* END *>
\end{verbatim}

