% !!! XRef for operation modes

\chapter{Using the compiler}\label{usage}

\section{Invoking the compiler}

The \xds{} \mt{} and \ot{} compilers are combined together with
the make subsystem and an \ot{} browser into a single
utility, \xc{}. When invoked without parameters, the utility outputs
a brief help information.

\xc{} is invoked from the command line of the following form
\index{\XC{}\Exe{}@{\XC{}\Exe{}}}

\verb'    '\xc{}\verb' { mode | option | name }'

where \verb'name', depending on the operation \verb'mode' can be
a module name, a source file name, or a project file name.
See \ref{xc:modes} for a full description of operation modes.

\verb'option' is a compiler setup directive (See \ref{config:options}).
All options are applied to all operands, notwithstanding their relative order
on the command line. On some platforms, it may be
necessary to enclose setup directives in quotation marks:

\verb'    '\xc{}\verb" hello.mod '-checkindex+'"

See Chapter \ref{options} for the list of all compiler options
and equations.

\subsection{Precedence of compiler options}\label{precedence}
\index{option precedence}
\index{precedence of options}

The \xc{} utility receives its options in the following order:
\begin{enumerate}
\item         from a configuration file {\bf \cfg} (See \ref{config:cfg})
\item         from the command line (See \ref{xc:modes})
\item         from a project file (if present) (See \ref{xc:project})
\item         from a source text (not all options can be used there)
              (See \ref{m2:pragmas})
\end{enumerate}

At any point during operation, the last value of an option is
in effect. Thus, if the equation \OERef{OBERON} was set to {\bf.ob2} in a
configuration file, but then set to {\bf.o2} on the command line, the
compiler will use {\bf.o2} as the default \ot{} extension.

\section{\XDS{} compilers operation modes}\label{xc:modes}
\index{operation modes}

\xds{} \mt{}/\ot{} compilers have the following operation modes:
\begin{tabular}{ll}
\bf Mode    & \bf Meaning                                     \\
\hline
\TRef{COMPILE}{xc:modes:compile} & Compile all modules given on the command line  \\
\TRef{PROJECT}{xc:modes:project} & Make all projects given on the command line    \\
\TRef{MAKE}{xc:modes:make}       & Check dependencies and recompile               \\
\TRef{GEN}{xc:modes:gen}         & Generate makefile for all projects             \\
\TRef{BROWSE}{xc:modes:browse}   & Extract definitions from symbol files          \\
HELP                            & Print help and terminate
\end{tabular}

Both the PROJECT and MAKE modes have two optional operation submodes:
\See{BATCH}{}{xc:modes:batch} and \See{ALL}{}{xc:modes:all}.
Two auxiliary operation submodes --- \See{options}{}{xc:modes:options}
and \See{EQUATIONS}{}{xc:modes:equations} can be used to inspect the set of
compiler options and equations and their values.

On the command line, the compiler mode is specified with the "\verb'='" symbol
followed by a mode name. Mode names are not case sensitive, and specifying
an unique portion of a mode name is sufficient, thus

\verb'    =PROJECT' is equivalent to   \verb'=p' \\
\verb'    =BROWSE ' is equivalent to   \verb'=Bro'

Operation modes and options can be placed on the command line in arbitrary
order, so the following two command lines are equivalent:

\verb'    '\xc{}\verb' =make hello.mod =all -checknil+' \\
\verb'    '\xc{}\verb' -checknil+ =a =make hello.mod'

\subsection{COMPILE mode}\label{xc:modes:compile}
\index{operation modes!COMPILE}

\verb'    '\xc{}\verb' [=compile] { FILENAME | OPTION }'

COMPILE is the default mode, and can be invoked simply by  supplying
\xc{} with a source module(s) to compile. If \xc{} is invoked without a
given  mode, COMPILE mode is assumed. In order to determine which
compiler should be used, \xc{} looks at the extensions of the given source
files. The default mapping of extensions is given below :
\begin{flushleft}
\begin{tabular}{lcl}
\bf        .mod  &-& Modula-2 implementation module \\
\bf        .def  &-& Modula-2 definition module     \\
\bf        .ob2  &-& Oberon-2 module                \\
\end{tabular}
\end{flushleft}

For example:

\verb'    '\xc{}\verb' hello.mod'

will invoke the \mt{} compiler, whereas:

\verb'    '\xc{}\verb' hello.ob2'

will invoke the \ot{} compiler.

The user is able to reconfigure  the  extension  mapping
(See \ref{config:fileext}). It is also
possible to override it from the command line
using the options \OERef{M2} and \OERef{O2}:

\verb'    '\xc{}\verb' hello.mod +o2  (* invokes O2 compiler *)'  \\
\verb'    '\xc{}\verb' hello.ob2 +m2  (* invokes M2 compiler *)'

{\bf Note:} In the rest of this manual, the COMPILE mode also
refers to any case in which the compiler {\it compiles}
a source file, regardless of the actually specified mode
(which can be COMPILE, MAKE, or PROJECT).
For instance, an option or equation, which is stated to affect
the compiler behaviour in the COMPILE mode, is relevant to
MAKE and PROJECT modes as well.

\subsection{MAKE mode}\label{xc:modes:make}
\index{operation modes!MAKE}

\verb'    '\xc{}\verb' =make [=batch] [=all] { FILENAME | OPTION }'

In the MAKE mode the compiler determines module dependencies using
\verb'IMPORT' clauses and then recompiles all necessary modules.
Starting from the files on the command line, it tries to find an \ot{}
module or a definition and implementaion module for each imported
module. It then does the same for each of the imported modules until
all modules are located. Note that a search is made for source files
only. If a source file is not found, the imported modules will not be
appended to the recompile list. See section \ref{xc:make} for more details.

When all modules are gathered, the compiler performs an action according to
the operation submode. If the \See{BATCH submode}{}{xc:modes:batch} was specified, it
creates a batch file of all necessary compilations, rather than
actually compiling the source code.

If the \See{ALL submode}{}{xc:modes:all} was specified, all gathered files are
recompiled, otherwise \xds{} recompiles only the necessary files.
The {\em smart recompilation} algorithm is described in
\ref{xc:smart}.

Usually, a \mt{} program module or an \ot{} top-level module
is specified on the command line. In this case, if the \OERef{LINK} equation
is set in either configuration file or \xc{} command line,
the linker will be invoked automatically in case of successful
compilation. This feature allows you to build simple programs without
creating project files.

\subsection{PROJECT mode}\label{xc:modes:project}
\index{operation modes!PROJECT}

\verb'    '\xc{}\verb' =project [=batch] [=all] { PROJECTFILE | OPTION }'

The PROJECT mode is essentially the same as the MAKE mode except that the
modules to be `made' are provided in a project file. A project
file specifies a set of options and a list of modules.
See \ref{xc:project} for further details.
As in the MAKE mode, \See{ALL}{}{xc:modes:all} and \See{BATCH}{}{xc:modes:batch}
submodes can be used.

If a file extension of a project file is omitted, \xds{} will use
an extension given by the equation \OERef{PRJEXT}
({\bf .prj} by default). \index{.prj (See PRJEXT)}

It may be necessary to compile a single module in the environment
specified in a project file. It can be accomplished in the COMPILE
operation mode using with the \OERef{PRJ} equation:

\verb'    '\xc{}\verb' -prj=myproject MyModule.mod'

\Seealso
\begin{itemize}
\item   MAKE operation mode: \ref{xc:modes:make}
\item   Make strategy: \ref{xc:make}
\item   Smart recompilation: \ref{xc:smart}
\end{itemize}

\subsection{GEN mode}\label{xc:modes:gen}
\index{operation modes!GEN}
\index{template files}

\verb'    '\xc{}\verb' =gen { PROJECTFILE | OPTION }'

The  GEN operation mode allows one to generate a file containing
information about your project. The most important usage is to
generate
\ifgenc
  a makefile, which can then be passed to the make utility
  accompanying the "underlaying" C compiler, so that all
  generated C files can be compiled and linked into an
  executable program.
\else
  a linker response file (See \ref{start:build}).
\fi

This operation mode can also be used to obtain additional
information about your project, e.g. a list of all
modules, import lists, etc.

A so-called template file, specified by the \OERef{TEMPLATE} equation,
is used in this mode. A template file is a text file, some lines of which are
marked with a certain symbol. All the lines which are not marked are
copied to the output file verbatim. The marked lines are processed
in a special way. See \ref{xc:template} for more information.

The compiler creates a file with a name specified by
the equation \OERef{MKFNAME}. If the equation is empty,
the project file name is used.
A file name is then concatenated with the
extension specified by the equation \OERef{MKFEXT}.
\index{.mkf (See MKFEXT)}

\subsection{BROWSE mode}\label{xc:modes:browse}
\index{operation modes!BROWSE}

\verb'    '\xc{}\verb' =browse { MODULENAME | OPTION }'

The BROWSE operation mode allows one to generate a pseudo definition
module for an \ot{} module. In this mode, the compiler reads a symbol
file and produces a file which contains
declarations of all objects exported from the \ot{} module,
if a format resembling \mt{} definition modules.

The configuration option \OERef{BSDEF}
specifies the extension of a generated file.
If this option is not set, then the default extension ({\bf .odf})
will be used. \index{.odf (See BSDEF)}

Options \OERef{BSCLOSURE} and \OERef{BSREDEFINE} can be used to control the
form of a generated file. {\bf Note:} the \OERef{BSTYLE} equation
(described in \ref{o2:env:makedef}) is ignored in this operation mode,
and the browse style is always set to DEF.

The \OERef{MAKEDEF} option (See \ref{o2:env:makedef}) provides
an alternative method of producing pseudo definition modules,
preserving so-called {\em exported} comments if necessary.

\subsection{ALL submode}
\label{xc:modes:all}
\index{operation modes!ALL}

In both \TRef{PROJECT}{xc:modes:project} and \TRef{MAKE}{xc:modes:make} modes,
the compiler checks the time stamps of the files concerned and recompiles
only those files that are necessary (See \ref{xc:smart}).
If the ALL submode was specified, the time stamps are ignored,
and all files are compiled.

\subsection{BATCH submode}\label{xc:modes:batch}
\index{operation modes!BATCH}

In the BATCH submode, the compiler creates a batch file of all necessary
compilations, rather than actually calling the compilers and compiling
the source code.

A batch file is a sequence of lines beginning with the compiler name,
followed by module names to recompile.

The compiler creates a batch file with a name determined by either:
\begin{enumerate}
\item The compiler option \OERef{BATNAME}
\item The project file name (if given)
\item The name {\bf out} (if the name could not be determined
      by the above).
\end{enumerate}

The name is then concatenated  with the batch  file
extension specified by the equation \OERef{BATEXT}
({\bf.bat} by default).
\index{.bat (See BATEXT)}

\Seealso
\begin{itemize}
\item option \OERef{LONGNAME}   (\ref{opt:bool})
\item equation \OERef{BATWIDTH} (\ref{opt:equ})
\end{itemize}

\subsection{OPTIONS submode}\label{xc:modes:options}
\index{operation modes!OPTIONS}

The OPTIONS submode allows you to inspect the values of options which
are set in the configuration file, project file and on the command
line. It can be used together with
\See{COMPILE}{}{xc:modes:compile},
\See{MAKE}{}{xc:modes:make}, and
\See{PROJECT}{}{xc:modes:project} modes.

The following command line prints (to the standard output) the
list of all defined options, including all pre-declared options,
all options declared in the configuration file, in the project
file {\tt my.prj} and on the command line ({\tt xyz} option):

\verb'    '\xc{}\verb' =options -prj=my.prj -xyz:+'

In the PROJECT mode options are listed for each project file
given on the command line.

See also the \TRef{EQUATIONS submode}{xc:modes:equations}.

\subsection{EQUATIONS submode}\label{xc:modes:equations}
\index{operation modes!EQUATIONS}

The EQUATIONS submode allows you to inspect the values of equations
which are set in the configuration file, project file and on the command
line.
It can be used together with
\See{COMPILE}{}{xc:modes:compile},
\See{MAKE}{}{xc:modes:make}, and
\See{PROJECT}{}{xc:modes:project} modes.

See also the \TRef{OPTIONS submode}{xc:modes:options}.

\section{Files generated during compilation}\label{usage:genfiles}

When applied to a file which contains a module {\bf name},
the compilers produce the following files.

\subsection{Modula-2 compiler}

When applied to a definition module, the \mt{} compiler produces a
{\em symbol file}\index{symbol files}\index{\Sym (See SYM)}
\ifgenc
  ({\bf name\Sym}) and a C header file ({\bf name\Header}).
  \index{\Header (See HEADER)}
  Generation of a header file can be prevented by use of
  the \OERef{NOHEADER} option.
\else
  ({\bf name\Sym}).
\fi
The symbol file contains information required
during compilation of a module which imports the module {\bf name}.

When applied to an implementation module or a top level module,
the \mt{} compiler produces
\ifgencode an object \fi
\ifgenc a C code    \fi
file ({\bf name\Code}).\index{\Code (See CODE)}

\subsection{Oberon-2 compiler}

For all compiled modules, the Oberon-2 compiler
produces
\ifgencode
  a {\em symbol file} ({\bf name\Sym})
  and an object file ({\bf name\Code}).
\fi
\ifgenc
  a {\em symbol file} ({\bf name\Sym}),
  a C header file ({\bf name\Header}) and a code file ({\bf name\Code}).
\fi
The symbol file\index{symbol files}
({\bf name\Sym}) contains information required
during compilation of a module which imports the module {\bf name}.
If the compiler needs to overwrite an existing symbol file,
it will only do so if the \OERef{CHANGESYM} option is set ON.

\Examples
\begin{tabular}{ll}
\bf Command line                 & \bf Generated files \\
\hline
\tt \xc{} Example.def            & \tt Example\Sym  \\
\ifgenc
                                 & \tt Example\Header \\
\fi
\tt \xc{} Example.mod            & \tt Example\Code  \\
\tt \xc{} Win.ob2 +CHANGESYM     & \tt Win\Sym  \\
\ifgenc
                                 & \tt Win\Header  \\
\fi
                                 & \tt Win\Code \\
\end{tabular}

\section{Control file preprocessing}
\label{usage:cfp}

An \XDS{} compiler may read the following control files during execution:
\begin{itemize}
\item a \See{redirection file}{}{xc:red}
\item a \See{configuration file}{}{config:cfg}
\item a \See{project file}{}{xc:project}
\item a \See{template file}{}{xc:template}
\end{itemize}
All these files are preprocessed during read according to the
following rules:

A control file is a plain text file containing a sequence of lines.
The backslash character (\verb|"\"|) at the end of a line denotes its
continuation.

The following constructs are handled during control file preprocessing:
\begin{itemize}
\item macros of the kind \verb'$('{\it name}\verb')'. A macro expands
      to the value of the equation {\it name} or, if it does not exist,
      to the value of the environment variable {\it name}.
\item the {\em base directory} macro (\verb|$!|) % \index{$!@\verb'$!'}. !!!
      This macro expands to the directory in which the file containing
      it resides.
\item a set of directives, denoted by the exclamation mark (\verb|"!"|)
      as a first non-whitespace character on a line.
\end{itemize}

A directive has the following syntax (all keywords are case independent):
\begin{verbatim}
Directive = "!" "NEW" SetOption | SetEquation
          | "!" "SET" SetOption | SetEquation
          | "!" "MESSAGE" Expression
          | "!" "IF" Expression "THEN"
          | "!" "ELSIF" Expression "THEN"
          | "!" "ELSE"
          | "!" "END".
SetOption   = name ( "+" | "-" ).
SetEquation = name "=" string.
\end{verbatim}

The \verb'NEW' directive declares a new option or equation. The \verb'SET' directive
changes the value of an existent option or equation. The \verb'MESSAGE' directive
prints \verb'Expression' value to the standard output.
The \verb'IF' directive allows to process or skip portions of files according
to the value of \verb'Expression'. \verb'IF' directives may be nested.

\begin{verbatim}
Expression  = Simple [ Relation Simple ].
Simple      = Term { "+" | OR Term }.
Relation    = "=" | "#" | "<" | ">".
Term        = Factor { AND Factor }.
Factor      = "(" Expression ")".
            | String
            | NOT Factor
            | DEFINED name
            | name.
String      = "'" { character } "'"
            | '"' { character } '"'.
\end{verbatim}

An operand in an expression is either string, equation name, or option
name. In the case of equation, the value of equation is used. In the
case of option, a string \verb'"TRUE"' or \verb'"FALSE"' is used.
The \verb'"+"' operator denotes string concatenation.
Relation operators perform case insensitive string comparison.
The \verb'NOT' operator may be applied to a string with value
\verb'"TRUE"' or \verb'"FALSE"'. The \verb'DEFINED' operator yields \verb'"TRUE"' if
an option or equation \verb'name' is declared and \verb'"FALSE"' otherwise.

See also section \ref{opt:COMPILER}.

\section{Project files}\label{xc:project}
\index{project files}

A project file has the following structure:

\verb'    {SetupDirective}' \\
\verb'    {!module {FileName}}'

Setup directives define options and equations that all modules
which constitute the project should be compiled with.
See also \ref{config:options} and \ref{usage:cfp}.

Every line in a project file can contain only one setup directive. The
character "\verb|%|" indicated a comment; it causes the rest of
a line to be discarded. {\bf Note:} the comment character can not be
used in a string containing equation setting.

Each  {\tt FileName} is a name of a file which should be compiled,
linked, or otherwise processed when a project is being built, e.g.
a source file, an additional library, a resource file (on Windows),
etc. The compiler processes only \mt{} and \ot{}
source files. The type of a file is determined by its extension
(by default \mt{}/\ot{} source files extension is assumed).
Files of other types are taken into account only when a template file
is processed (see \ref{xc:template}).

The compiler recursively scans import lists of all
specified \mt{}/\ot{} source modules and builds the full list of modules used
in the project. Thus, usually, a project file for an executable program
would contain a single {\tt !module} directive for the file which
contains the main program module and, optionally, several {\tt !module}
directives for non-source files.

At least one \verb'!module' directive should be specified in a project file.

A project file can contain several \OERef{LOOKUP} equations, which
allow you to define additional search paths.

\xds{} compilers give you complete freedom over where to set options, equations
and redirection directives. However, it is recommended to specify
only those settings in the configuration and redirection files
which are applied to all your projects, and use project files
for all project-specific options and redirection directives.

\begin{figure}[htb]
\begin{verbatim}
    -lookup = *.mod = mod
    -lookup = *.sym = sym; $(XDSDIR)/sym/C
    % check project mode
    !if not defined mode then
      % by default use debug mode
      !new mode = debug
    !end
    % report the project mode
    !message "Making project in the " + mode + " mode"
    % set options according to the mode
    !if mode = debug then
       - gendebug+
       - checkrange+
    !else
       - gendebug-
    !fi
    % specify template file
    - template = $!/templates/watcom.tem
    !module hello
    !module hello.res
\end{verbatim}
\caption{A Sample Project File}
\label{xc:project:sample}
\end{figure}

Given the sample project file shown on Figure \ref{xc:project:sample},
the compiler will search for files with \verb'.mod'
and \verb'.sym' extensions using search paths specified in the
project file {\it before} paths specified in a redirection file.

A project file is specified explicitly in the \See{PROJECT}{}{xc:modes:project}
and \See{GEN}{}{xc:modes:gen} operation modes. In these modes, all
options and equations are set and then the compiler proceeds through
the module list to gather all modules constituting a project (See \ref{xc:make}).

In the \See{MAKE}{}{xc:modes:make} and \See{COMPILE}{}{xc:modes:compile}
operation modes, a project file can be specified using the \OERef{PRJ}
equation. In this case, the module list is ignored, but all options
and equations from the project file are set.

The following command line forces the compiler
to compile the module {\tt hello.mod} using options
and equations specified in the
project file {\tt hello.prj}:

\verb'    '\xc{}\verb' hello.mod -prj=hello.prj'

\section{Make strategy}\label{xc:make}

This section concerns
\See{MAKE}{}{xc:modes:make},
\See{PROJECT}{}{xc:modes:project}, and
\See{GEN}{}{xc:modes:gen},
operation modes.
In these modes, an \XDS{} compiler builds a set of all modules that
constitute the project, starting from the modules specified in a
project file (PROJECT and GEN) or on the command line (MAKE).

The MAKE mode is used in the following examples, but
the comments also apply to the PROJECT and GEN modes.

First, the compiler tries to find all given modules according to the
following strategy:
\begin{itemize}
\item
        If both filename extension and path are present,
        the compiler checks if the given file exists.

        \verb'    '\xc{}\verb' =make mod'\DirSep{}\verb'hello.mod'

\item
        If only an extension is specified, the compiler seeks
        the given file using search paths.

        \verb'    '\xc{}\verb' =make hello.mod'

\item
        If no extension is specified, the compiler searches for
        files with the given name and the \ot{} module extension, \mt{}
        implementation module extension, and \mt{} definition
        module extension.

        \verb'    '\xc{}\verb' =make hello'

        An error is raised if more than one file was found,
        e.g. if both {\tt hello.ob2} and {\tt hello.mod} files exist.
\end{itemize}

Starting from the given files, the compiler tries to find an \ot{}
source module or \mt{} definition and implementation modules for
each imported module. It then tries to do the same for each of
the imported modules until all the possible modules are located.
For each module, the compiler checks correspondence between the
file name extension and the kind of the module.

% impext.tex !!!

\section{Smart recompilation}\label{xc:smart}

In the \See{MAKE}{}{xc:modes:make} and \See{PROJECT}{}{xc:modes:project}
modes, if the \See{ALL}{}{xc:modes:all} submode was not specified,
an \XDS{} compiler performs {\em smart recompilation} of modules which are
inconsistent with the available source code files. The complier
uses file modification time to determine which file has been changed.
For each module the decision (to recompile or not) is made
only after the decision is made for all modules on which it depends.
A source file is (re)compiled if one or more of the following conditions is true:
\begin{description}
\item[\mt{} definition module] \mbox{}

        \begin{itemize}
        \item the symbol file is missing
        \item the symbol file is present but
          its modification date is earlier than that of the
          source file or one of the imported symbol files
        \ifgenc
          \item the header file is missing (\OERef{NOHEADER} option is OFF)
          or its modification date is earlier than that of the source file
        \fi
\ifcomment
        \item the source file modification date is earlier than that
        of the project file (in the PROJECT mode only)
\fi
        \end{itemize}

\item[\mt{} implementation module] \mbox{}

        \begin{itemize}
        \item the code file is missing
        \item the code file is present but
          the file modification date is earlier than that of the
          source file or one of the imported symbol files
          (including its own symbol file)
\ifcomment
        \item the source file modification date is earlier than that
        of the project file (in the PROJECT mode only)
\fi
        \end{itemize}

\item[\mt{} program module] \mbox{}

        \begin{itemize}
        \item the code file is missing
        \item the code file is present but
          the file modification date is earlier than that of the
          source file or one of the imported symbol files
\ifcomment
        \item the source file modification date is earlier than that
        of the project file (in the PROJECT mode only)
\fi
        \end{itemize}

\item[Oberon-2 module] \mbox{}

        \begin{itemize}
        \item the symbol file is missing
        \item the symbol file is present but
          the modification date is earlier than that
          of one of the imported symbol files
        \ifgenc
          \item the header file is missing (\OERef{NOHEADER} option is OFF)
        \fi
        \item the code file is missing
        \item the code file is present but
          the file modification date is earlier than that of the
          source file or one of the imported symbol files
\ifcomment
        \item the source file modification date is earlier than that
        of the project file (in the PROJECT mode only)
\fi
        \end{itemize}
\end{description}

When the \OERef{VERBOSE} option is set ON, the compiler reports a reason
for recompilation of each module.  {\bf Note:} if an error occured during
compilation of a \mt{} definition module or an \ot{} module,
all its client modules are not compiled at all.

\section{Template files}\label{xc:template}
\index{template files}

A {\it template file} is used to build a "makefile"
in the \See{PROJECT}{}{xc:modes:project}
and \See{GEN}{}{xc:modes:gen} operation modes,
if the option \OERef{MAKEFILE} is ON\footnote{
"MAKEFILE" is a historical name; a linker or library manager response file
may be built as well.}.

The compiler copies lines from a template file into the
output file verbatim, except lines marked as requiring further
attention. A single character (attention mark) is specified by the
equation \OERef{ATTENTION} (default is '!') % \index{! (See ATTENTION)}. % !!! ! in \index{}

A template file is also subject to \See{preprocessing}{}{usage:cfp}.

A marked line (or template) has the following
format\footnote{The same syntax is used in the
\OERef{LINK} equation.}:
\begin{verbatim}
    Template  = { Sentence }.
    Sentence  = Item { "," Item } ";" | Iterator.
    Item      = Atom | [ Atom | "^" ] "#" [ Extension ].
    Atom      = String | name.
    String    = '"' { character } '"'
              | "'" { character } "'".
    Extension = [ ">" ] Atom.
    Iterator  = "{" Set ":" { Sentence } "}".
    Set       = { Keyword | String }
    Keyword   = DEF | IMP | OBERON | MAIN
              | C | HEADER | ASM | OBJ.
\end{verbatim}
{\tt name} should be a name of an equation. Not more than three items
may be used in a sentence. A first item in a sentence is a format
string, while others are arguments.                                        % !!! What it may contain?

The \XDS{} distribution contains a template file {\tt \xc{}.tem} which
can be used to produce
\ifgencode
a linker response file.
\fi
\ifgenc
a makefile for one of the supported C compilers.
\fi

\subsection{Using equation values}

In the simplest form, a template line may be used to output a value of
an equation. For example, if the template file contains the line

\verb'     ! "The current project is %s.\n",prj;'

and the project {\tt prj\DirSep{}test.prj} is processed,
the output will contain the line

\verb'    The current project is prj'{\tt \DirSep{}}\verb'test.prj.'

{\bf Note:} the line

\verb'    ! prj;'

is valid, but may produce unexpected results under systems in which
the backslash character ("\verb'\'") is used as a directory names separator
(e.g. OS/2 or Windows):

\verb'    prj     est.prj'

because \verb'"\t"' in a format string is replaced with the tab character.
Use the following form instead:

\verb'    ! "%s",prj;'

\subsection{File name construction}\label{xc:template:fname}

The \verb|"#"| operator constructs a file name from a name and an
extension, each specified as an equation name or literal string.
A file is then searched for according to \XDS{} search paths
and the resulting name is substituted.
For example, if the file \verb'useful.lib' resides in the directory '../mylibs'
and the redirection file contains the following line:

\verb'    *.lib = /xds/lib;../mylibs'

the line

\verb'    ! "useful"#"lib"'

will produce

\verb'    ../mylibs/useful.lib'

If the modifier \verb|">"| is specified, the compiler assumes that the file
being constructed is an output file and creates its name according to the
strategy for output files (See \ref{xc:red} and the \OERef{OVERWRITE}
option).

The \verb'"#"' operator is also used to represent the current value of
an \See{iterator}{}{xc:template:iterators}.
The form in which a name or extension is omitted can be used
in an iterator only.

The form \verb'"^#"' may be used in a second level iterator to represent
the current value of the first level iterator.

\subsection{Iterators}
\label{xc:template:iterators}

{\em Iterators} are used to generate some text for all modules from a given
set. Sentences inside the first level of braces are repeated
for all modules of the project, while sentences inside the second level are
repeated for all modules imported into the module currently iterated at
the first level. A set is a sequence of keywords and strings.
Each string denotes a specific module, while a keyword denotes all
modules of specific kind.

The meaning of keywords is as follows:
\begin{tabular}{lp{8cm}}
\bf Keyword &\bf Meaning  \\
 \hline
\tt DEF     & Modula-2 definition module                       \\
\tt IMP     & Modula-2 implementation module                   \\
\tt MAIN    & Modula-2 program module or Oberon-2 module marked as \OERef{MAIN} \\
\tt OBERON  & Oberon module                                    \\
\ifgenc
\tt C       & C source text                                    \\
\tt HEADER  & C header file                                    \\
\fi
\tt ASM     & assembler source text                            \\
\tt OBJ     & object file                                      \\
%\tt PROJECT & project name                                    \\
\end{tabular}

% There are other keywords !!!

A keyword not listed above is treated as filename extension.
Sentences are repeated for all files with that extension which are
explicitly specified in the project file using \verb'!module' directives
(see \ref{xc:project}).
This allows, for instance, additional
libraries to be specified in a project file:

\begin{verbatim}
sample.prj:

    -template = mytem.tem
    !module Sample.mod
    !module mylib.lib

mytem.tem:
      .  .  .
    ! "%s","libxds"#"lib"
    ! { lib: "+%s",#; }
    ! "\n"
      .  .  .

generated file:
      .  .  .
    d:\xds\lib\x86\libxds.lib+mylib.lib
      .  .  .
\end{verbatim}

\subsection{Examples}

Consider a sample project which consists of a program module {\tt A},
which imports modules {\tt B} and {\tt C}, and
{\tt B}, in turn, imports {\tt D} (all modules are written in \mt{}):
\begin{verbatim}
                A
               / \
              B   C
              |
              D
\end{verbatim}
The following examples illustrate template files usage:

This template line lists all project modules for
which source files are available:

\verb'    ! { imp oberon main: "%s ",#; }'

For the sample project, it would generate the following line:

\verb'    A.mod B.mod C.mod D.mod'

To output both definition and implementation modules, the following
lines may be used:

\verb'    ! { def : "%s ",#; }'\\
\verb'    ! { imp oberon main: "%s ",#; }'

The output would be:

\verb'    B.def C.def D.def A.mod B.mod C.mod D.mod'

The last template line may be used to list all modules
along with their import:

\verb'    ! { imp main: "%s\n",#; { def: "  %s\n",#; } }'

The output:

\begin{verbatim}
    A.mod
      B.def
      C.def
    B.mod
      D.def
    C.mod
    D.mod
\end{verbatim}
